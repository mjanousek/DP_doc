%==============================================================================
% tento soubor pouzijte jako zaklad
% this file should be used as a base for the thesis
% (c) 2008 Michal Bidlo
% E-mail: bidlom AT fit vutbr cz
% Šablonu upravil / template edited by: Ing. Jaroslav Dytrych, dytrych@fit.vutbr.cz
%==============================================================================
% kodovaní: UTF-8 (zmena prikazem iconv, recode nebo cstocs)
% encoding: UTF-8 (you can change it by command iconv, recode or cstocs)
%------------------------------------------------------------------------------
% zpracování / processing: make, make pdf, make clean
%==============================================================================
% Soubory, které je nutné upravit: / Files which have to be edited:
%   xjanou14-dynamicke-analyzatory-20-literatura-bibliography.bib - literatura / bibliography
%   xjanou14-dynamicke-analyzatory-01-kapitoly-chapters.tex - obsah práce / the thesis content
%   xjanou14-dynamicke-analyzatory-30-prilohy-appendices.tex - přílohy / appendices
%==============================================================================
%\documentclass[]{fitthesis} % bez zadání - pro začátek práce, aby nebyl problém s překladem
%\documentclass[english]{fitthesis} % without assignment - for the work start to avoid compilation problem
\documentclass[print]{fitthesis} % odevzdani do wisu - odkazy jsou barevné
%\documentclass[english,zadani]{fitthesis} % for submission to the IS FIT - links are color
%\documentclass[zadani,print]{fitthesis} % pro tisk - odkazy jsou černé
%\documentclass[english,zadani,print]{fitthesis} % for the print - links are black
% * Je-li prace psana v anglickem jazyce, je zapotrebi u tridy pouzit 
%   parametr english nasledovne:
%   If thesis is written in english, it is necessary to use 
%   parameter english as follows:
%      \documentclass[english]{fitthesis}
% * Je-li prace psana ve slovenskem jazyce, je zapotrebi u tridy pouzit 
%   parametr slovak nasledovne:
%      \documentclass[slovak]{fitthesis}

% Základní balíčky jsou dole v souboru šablony fitthesis.cls
% Basic packages are at the bottom of template file fitthesis.cls
%zde muzeme vlozit vlastni balicky / you can place own packages here

%---rm---------------
\renewcommand{\rmdefault}{lmr}%zavede Latin Modern Roman jako rm / set Latin Modern Roman as rm
%---sf---------------
\renewcommand{\sfdefault}{qhv}%zavede TeX Gyre Heros jako sf
%---tt------------
\renewcommand{\ttdefault}{lmtt}% zavede Latin Modern tt jako tt

% vypne funkci šablony, která automaticky nahrazuje uvozovky,
% aby nebyly prováděny nevhodné náhrady v popisech API apod.
% disables function of the template which replaces quotation marks
% to avoid unnecessary replacements in the API descriptions etc.
\csdoublequotesoff

% =======================================================================
% balíček "hyperref" vytváří klikací odkazy v pdf, pokud tedy použijeme pdflatex
% problém je, že balíček hyperref musí být uveden jako poslední, takže nemůže
% být v šabloně
% "hyperref" package create clickable links in pdf if you are using pdflatex.
% Problem is that this package have to be introduced as the last one so it 
% can not be placed in the template file.
\ifWis
\ifx\pdfoutput\undefined % nejedeme pod pdflatexem / we are not using pdflatex
\else
  \usepackage{color}
  \usepackage[unicode,colorlinks,hyperindex,plainpages=false,pdftex]{hyperref}
  \definecolor{links}{rgb}{0.4,0.5,0}
  \definecolor{anchors}{rgb}{1,0,0}
  \def\AnchorColor{anchors}
  \def\LinkColor{links}
  \def\pdfBorderAttrs{/Border [0 0 0] }  % bez okrajů kolem odkazů / without margins around links
  \pdfcompresslevel=9
\fi
\else % pro tisk budou odkazy, na které se dá klikat, černé / for the print clickable links will be black
\ifx\pdfoutput\undefined % nejedeme pod pdflatexem / we are not using pdflatex
\else
  \usepackage{color}
  \usepackage[unicode,colorlinks,hyperindex,plainpages=false,pdftex,urlcolor=black,linkcolor=black,citecolor=black]{hyperref}
  \definecolor{links}{rgb}{0,0,0}
  \definecolor{anchors}{rgb}{0,0,0}
  \def\AnchorColor{anchors}
  \def\LinkColor{links}
  \def\pdfBorderAttrs{/Border [0 0 0] } % bez okrajů kolem odkazů / without margins around links
  \pdfcompresslevel=9
\fi
\fi
% Řešení problému, kdy klikací odkazy na obrázky vedou za obrázek
% This solves the problems with links which leads after the picture
\usepackage[all]{hypcap}

%%%%%%%%%%%%%%%%%%%%%%%%%%%%%%%%%%%%%%%%%%%%%%%%%%%%%%%%%%%%%%%%
%%%%%%%%%%%%%%%%%%%%%%%%%%%%%%%%%%%%%%%%%%%%%%%%%%%%%%%%%%%%%%%%
%moje dalsi balicky
%%%%%%%%%%%%%%%%%%%%%%%%%%%%%%%%%%%%%%%%%%%%%%%%%%%%%%%%%%%%%%%%
%%%%%%%%%%%%%%%%%%%%%%%%%%%%%%%%%%%%%%%%%%%%%%%%%%%%%%%%%%%%%%%%
\usepackage{amssymb}
\usepackage{amsmath}
\usepackage{txfonts}
\usepackage{rotating}
\usepackage{centernot}
\makeatletter
\newcommand{\strokearrow}{\mathrel{\mathpalette\stroke@arrow{/}}}
\newcommand{\dstrokearrow}{\mathrel{\mathpalette\stroke@arrow{/\!\!/}}}
\newcommand{\stroke@arrow}[2]{%
  \ooalign{\hidewidth$\m@th#1#2$\hidewidth\cr$\m@th#1\longrightarrow$\cr}%
  \vphantom{#2}
}
\makeatother
\usepackage{listings}
\usepackage{parcolumns}
\lstset{language=Java,
  basicstyle=\footnotesize\tt,
  showspaces=false,
  showtabs=false,
  breaklines=true,
  showstringspaces=false,
  breakatwhitespace=true%,
  %commentstyle=\color{pgreen},
  %keywordstyle=\color{pblue},
  %stringstyle=\color{pred},
  %basicstyle=\ttfamily,
  %moredelim=[il][\textcolor{pgrey}]{$$},
  %moredelim=[is][\textcolor{pgrey}]{\%\%}{\%\%}
}
\let\openbox\relax
\usepackage{amsthm}

\theoremstyle{plain}
\newtheorem{thm}{Theorem}[chapter] % reset theorem numbering for each chapter

\theoremstyle{definition}
\newtheorem{defn}[thm]{Definice} % definition numbers are dependent on theorem numbers
\newtheorem{exmp}[thm]{Příklad} % same for example numbers


% Informace o práci/projektu / Information about the thesis
%---------------------------------------------------------------------------
\projectinfo{
  %Prace / Thesis
  project=DP,            %typ prace BP/SP/DP/DR  / thesis type (SP = term project)
  year=2017,             %rok odevzdání / year of submission
  date=\today,           %datum odevzdani / submission date
  %Nazev prace / thesis title
  title.cs={Dynamické analyzátory pro platformu \\SearchBestie},  %nazev prace v cestine ci slovenstine (dle zadani) / thesis title in czech language (according to assignment)
  title.en={Dynamic Analyzers for SearchBestie Platform}, %nazev prace v anglictine / thesis title in english
  %Autor / Author
  author={Martin Janoušek},   %cele jmeno a prijmeni autora / full name and surname of the author
  author.name={Martin},   %jmeno autora (pro citaci) / author name (for reference) 
  author.surname={Janoušek},   %prijmeni autora (pro citaci) / author surname (for reference) 
  author.title.p=Bc., %titul pred jmenem (nepovinne) / title before the name (optional)
  %author.title.a=PhD, %titul za jmenem (nepovinne) / title after the name (optional)
  %Ustav / Department
  department=UITS, % doplnte prislusnou zkratku dle ustavu na zadani: UPSY/UIFS/UITS/UPGM
  %                  fill in appropriate abbreviation of the department according to assignment: UPSY/UIFS/UITS/UPGM
  %Skolitel / supervisor
  supervisor=Aleš Smrčka, %cele jmeno a prijmeni skolitele / full name and surname of the supervisor
  supervisor.name={Aleš},   %jmeno skolitele (pro citaci) / supervisor name (for reference) 
  supervisor.surname={Smrčka},   %prijmeni skolitele (pro citaci) / supervisor surname (for reference) 
  supervisor.title.p=Ing.,   %titul pred jmenem (nepovinne) / title before the name (optional)
  supervisor.title.a={Ph.D.},    %titul za jmenem (nepovinne) / title after the name (optional)
  %Klicova slova, abstrakty, prohlaseni a podekovani je mozne definovat 
  %bud pomoci nasledujicich parametru nebo pomoci vyhrazenych maker (viz dale)
  %Keywords, abstracts, declaration and acknowledgement can be defined by following 
  %parameters or using dedicated macros (see below)
  %===========================================================================
  %Klicova slova / keywords
  %keywords.cs={Klíčová slova v českém jazyce.}, %klicova slova v ceskem ci slovenskem jazyce
  %                                              keywords in czech or slovak language
  %keywords.en={Klíčová slova v anglickém jazyce.}, %klicova slova v anglickem jazyce / keywords in english
  %Abstract
  %abstract.cs={Výtah (abstrakt) práce v českém jazyce.}, % abstrakt v ceskem ci slovenskem jazyce
  %                                                         abstract in czech or slovak language
  %abstract.en={Výtah (abstrakt) práce v anglickém jazyce.}, % abstrakt v anglickem jazyce / abstract in english
  %Prohlaseni / Declaration
  %declaration={Prohlašuji, že jsem tuto bakalářskou práci vypracoval samostatně pod vedením pana ...},
  %Podekovani (nepovinne) / Acknowledgement (optional)
  %acknowledgment={Zde je možné uvést poděkování vedoucímu práce a těm, kteří poskytli odbornou pomoc.} % nepovinne
  %acknowledgment={Here it is possible to express thanks to the supervisor and to the people which provided professional help.} % optional
}

%Abstrakt (cesky, slovensky ci anglicky) / Abstract (in czech, slovak or english)
\abstract[cs]{Tato diplomová práce se zabývá problematikou testování paralelních programů. Popisuje chyby, které se mohou v paralelních programech nacházet a metody jejich testování. Podrobněji se zabývá metodou dynamické analýzy a věnuje se konkrétním dynamickým analyzátorům, jako jsou FastTrack nebo analyzátor kontraktů. Zabývá se také popisem nástrojů, které mohou být pro testování paralelních programů použity. Popsány jsou zejména nástroje SearchBestie a RoadRunner, pro které je dále popsán návrh jednoho z dynamických analyzátorů. Jedná se o návrh analyzátoru kontraktů.  }
\abstract[en]{This master thesis deals with testing of parallel programs. It discusses errors, which can be founded in parallel programs, and methods for their testing. It deals with dynamic analysis and particular dynamic analyzers, such as FastTrack or Contract analyzer. It describes platforms for testing of parallel programs, such as SearchBestie and RoadRunner. The thesis also presents a design of Contract analyzer, as a new component of RoadRunner and Searchestie platform.}

%Klicova slova (cesky, slovensky ci anglicky) / Keywords (in czech, slovak or english)
\keywords[cs]{Dynamická analýza, testování paralelních programů, analyzátor kontraktů, SearchBestie, RoadRunner}
\keywords[en]{Dynamic analysis, testing of parallel programs, contract analyzer, SearchBestie, RoadRunner}

%Prohlaseni (u anglicky psane prace anglicky, u slovensky psane prace slovensky)
%Declaration (for thesis in english should be in english)
\declaration{Prohlašuji, že jsem tuto diplomovou práci vypracoval samostatně pod vedením pana Ing. Aleše Smrčky Ph.D.
%Další informace mi poskytli...
Uvedl jsem všechny literární prameny a publikace, ze kterých jsem čerpal.}

% \declaration{Hereby I declare that this bachelor's thesis was prepared as an original author’s work under the supervision of Mr. X
% The supplementary information was provided by Mr. Y
% All the relevant information sources, which were used during preparation of this thesis, are properly cited and included in the list of references.}

%Podekovani (nepovinne, nejlepe v jazyce prace) / Acknowledgement (optional, ideally in the language of the thesis)
\acknowledgment{Rád bych poděkoval mému vedoucímu Ing. Aleši Smrčkovi Ph.D. za odborné vedení, za pomoc a~rady při zpracování této práce.
%V této sekci je možno uvést poděkování vedoucímu práce a těm, kteří poskytli odbornou pomoc(externí zadavatel, konzultant, apod.).
}
%\acknowledgment{Here it is possible to express thanks to the supervisor and to the people which provided professional help
%(external submitter, consultant, etc.).}

% řeší první/poslední řádek odstavce na předchozí/následující stránce
% solves first/last row of the paragraph on the previous/next page
\clubpenalty=10000
\widowpenalty=10000


\renewcommand{\labelenumii}{\theenumii}
\renewcommand{\theenumii}{\theenumi.\arabic{enumii}.}

\usepackage{syntax}

\begin{document}
  % Vysazeni titulnich stran / Typesetting of the title pages
  % ----------------------------------------------
  \maketitle
  % Obsah
  % ----------------------------------------------
  \tableofcontents
  
  % Seznam obrazku a tabulek (pokud prace obsahuje velke mnozstvi obrazku, tak se to hodi)
  % List of figures and list of tables (if the thesis contains a lot of pictures, it is good)
\ifczech
  \renewcommand\listfigurename{Seznam obrázků}
\fi
\ifslovak
  \renewcommand\listfigurename{Zoznam obrázkov}
\fi

  % \listoffigures
\ifczech
  \renewcommand\listtablename{Seznam tabulek}
\fi
\ifslovak
  \renewcommand\listtablename{Zoznam tabuliek}
\fi

  % \listoftables 

  % vynechani stranky v oboustrannem rezimu
  % Skip the page in the two-sided mode
  \iftwoside
    \cleardoublepage
  \fi

  % Text prace / Thesis text
  % ----------------------------------------------
  %=========================================================================
% (c) Michal Bidlo, Bohuslav Křena, 2008

\chapter{Úvod}

[TODO] S: motivace + aktualni stav + vlastni prinos
[TODO obsah]: motivace, proc to vzniklo, co existuje, jaky je aktualni stav, co se deje v kazde kapitole
 - idealne odstavec vlastni prinos, kde napisu co jsem konkretne udelal 

%Testování softwaru je důležitou součástí jeho vývoje. Cílem testování je nalézt chyby, které se v~testovaném softwaru vytvořily během jeho návrhu, implementace nebo provozování. Jedním z~typů chyb jsou chyby vyskytující se v~paralelních programech. Tyto chyby se typicky mohou vyskytovat náhodně a velmi omezeně, a proto odhalení tohoto typu chyb není snadné. Pro testování paralelních programů se využívají speciální techniky, jako je například \textit{dynamická analýza}. Dynamickou analýzu provádějí tzv. \textit{dynamické analyzátory}, které jsou předmětem této práce.

%V~této práci bude nejprve popsána problematika paralelních programů a chyb, které v~těchto programech mohou vznikat. Dále budou v~této práci popsány metody testování těchto programů, přičemž důraz bude kladen právě na dynamickou analýzu. Následně budou představeny nástroje, které lze pro takové testování použít, včetně nástroje \textit{SearchBestie}, pro který budou vybrané dynamické analyzátory implementovány. V~neposlední řadě budou popsány dynamické analyzátory, včetně dynamického analyzátoru \textit{kontraktů}, jehož návrh a implementace v~jazyce Java bude předmětem další práce.

\chapter{Testování vícevláknových programů}

[Poznamky z konzultace] technologie, metody v dane oblasti (testovani vicevlaknovych programu), jak se delaji dyn. analyz. v jave, roadrunner, searchbestie
technologie, se kterymi prichazim do styku; metody v dane oblasti (testovani vicevlaknovych programu - zhruba popsat tuto oblast, kontrakty, java)\\
1. jak se dela dynamicka analyza v jave\\
2. popsat co to jsou kontrakty\\
	+ popis roadrunneru a pouzitych technologii (searchbestie)\\
Mozna by to chtelo cast nasledujiciho textu presunout do uvodu.

V~dnešní době nalezneme v~běžných počítačích procesory, které mají 2 nebo více jader. Aby jeden paralelní program mohl využívat více těchto jader současně, musí využívat jeden z následujících způsobů. Prvním způsobem je použití více procesů. V~tomto kontextu lze proces chápat jako sekvenčně provádný samostatný program ve vlastním adresovém prostoru. %Nevýhodou tohoto řešení je rychlost přepínání kontextu a rychlost synchronizace mezi procesy. Typickým způsobem komunikace mezi vlákny je \textit{IPC} (Inter Process Communication)
\cite{cite:IPC, cite:paralel-programming}.

Druhým způsobem je použití \textit{vícevláknového} programu, ve kterém se nachází pouze jeden proces obsahující několik vláken\footnote{V jazyce Java je program chápán jako proces s~vlákny, přestože lze vytvořit i programy, které budou obsahovat více procesů. Nicméně každý proces obsahuje minimálně jedno vlákno \cite{cite:java}.}. V~tomto kontextu proces chápeme jako obálku sady souběžně prováděných vláken v~jednom adresovém prostoru. Přestože použití vláken přináší řadu výhod, jako například vyšší rychlost přepínání kontextu a sdílení prostředků, přináší také problémy v~podobě souběžného přístupu k těmto sdíleným prostředkům. Tento přístup musí být řízen pomocí tzv. \textit{synchronizačních prostředků}, jinak může docházet k chybám způsobených paralelizací. \cite{cite:paralel-programming}. 
%Rozdíl mezi výše zmíněnými způsoby je znázorněn na obrázku \ref{}.
%...obrazek...

Základním požadavkem v~dnešních operačních systémech\footnote{Dnešními operačními systémy jsou myšleny multiprogramové operační systémy, kde uživatel požaduje souběžný běh více programů a tedy souběžný běh více procesů a vláken.} je souběžné provádění více programů. Vzhledem k~tomu, že počet procesů a vláken, které mají běžet souběžně, je většinou vyšší než počet fyzických jader počítače, musí být výpočetní čas procesoru rozprostřen mezi všechny tyto procesy a vlákna. Rozprostření výpočetního času zajišťuje \textit{plánovač}, který plánuje přepínání kontextu procesů a vláken. Plánovač je typicky preemptivní\footnote{Při preemptivním plánování může být procesor procesu odebrán bez jeho přičinění (ukončení procesu, zahájení čekání, atd.).} a zohledňuje několik vlastností procesů a vláken, jako je například priorita, čas strávený na procesu, doba čekání nebo paměťové požadavky. Díky plánovači a dalším zdrojům nedeterminismu (obsluha přerušení, blokující operace, atd.) je paralelní provádění programu také nedeterministické, tj. procesy a vlákna mohou být různě \textit{proloženy}. Tento nedeterminismus způsobuje, že chyby vzniklé paralelním prováděním programu se v~jednom běhu mohou vyskytnout, zatímco v~jiném ne. Z~hlediska testování paralelních programů je tedy důležité ověřit, zda se chyba nevyskytuje v žádném možném proložení \cite{cite:OS1}.


%Jak to bude vzpadat:
%Testování vícevláknových programů\\
%(co to je)\\
%- soubeznost a relace HB\\
%- logicky cas\\
%- synchronizační nástroje\\
%- chyby v synchronizaci (popis vice chyb včetně kontraktu)\\
%- metody testování vícevláknových programů v jazyce Java (obecný popis všech metod v textu bez dalsich podnadpisu)\\
%- dynamické analyzátory (popis Fastracku, Contract validatoru)\\
%- nástroje pro testování (Searchbestie, Roadrunner)\\

%Co mam:\\
%Paralelni program\\
%Soubeznost a relace Happens Before\\
%Logicky cas\\
%Synchronizacni nastroje\\
%Chyby v synchronizaci\\

%Metody testovani paralelnich programu
%Staticka analyza
%Dynamicka analyza
%Deterministicke testovani
%Zatezove testovani
%Vkladani sumu

%Nastroje pro testovani paralelnich programu
%Searchbestie
%Roadrunner
%Dynamicke analyzatory - ??Fasttrack??, Analyzator kontraktu

\section{Souběžnost a relace Happens before}\label{happens-before}

V~paralelním programu je důležité určit pořadí jednotlivých událostí napříč procesy (vlákny). Protože výpočet jednotlivých procesů a vláken probíhá asynchronně, musí být zaveden způsob, jak pořadí těchto událostí určit. Vztah mezi dvěma událostmi, které se provedly ve stejném nebo různém procesu (vlákně) se nazývá \textit{relace kauzálního uspořádání} nebo taky relace \textit{Happens before} \cite{cite:HB}. Relace Happens before byla představena Lesliem Lamportem pro prostředí distribuovaných systémů, a~proto zde bude uvedena v~tomto obecnějším kontextu. Nicméně tato relace platí jak pro uspořádání událostí napříč vlákny, tak i procesy. V~této kapitole tedy budou uvažovány procesy a~vlákna za totožné. Následující vysvětlení této relace vychází ze zdrojů \cite{cite:HB,cite:PDI}.

%cite http://research.microsoft.com/en-us/um/people/lamport/pubs/time-clocks.pdf

Paralelní program se skládá z~množiny $n$ asynchronních procesů ${p_1, p_2, ..., p_n}$. Procesy spolu navzájem mohou komunikovat pouze pomocí zasílání zpráv\footnote{Pod zasílám zpráv je možné si představit také komunikaci pomocí sdílené paměti, přístupu k~zámkům, atd. Pro jednoduchost bude veškerá tato komunikace zobecněna jako zasílání zpráv mezi procesy.}. Dále platí, že procesy nemohou sdílet svůj \textit{globální čas\footnote{Globání čas procesu je pohled daného procesu na celkový čas (tj. čas všech procesů) v~paralelním programu. Každý proces může mít ve stejnou chvíli jiný pohled na celkový čas.}} a výpočet procesů, včetně zasílání zpráv, probíhá asynchronně (předpokládejme, že každý proces běží na vlastním. Zprávu zaslanou procesem $p_i$ procesu $p_j$ označme jako $m_{ij}$. Vykonávání procesu se skládá ze sekvence atomických událostí. Události mohou být tří typů \cite{cite:PDI}:

\begin{itemize}
\item interní akce -- mění vnitřní stav procesu,
\item odeslání zprávy $m$ (\textit{send($m$)}) -- mění vnitřní stav odesílajícího procesu,
\item přijetí zprávy $m$ (\textit{recv($m$)}) -- mění vnitřní stav přijímajícího procesu.
\end{itemize}

Dále nechť $e^x_i$ označuje $x$-tou událost procesu $i$. Pak platí, že události v~procesu jsou lineárně uspořádány podle indexu $x$. Toto lineární uspořádání událostí procesu $i$ označme jako relaci $\rightarrow_i$ a~množinu všech akcí procesu $p_i$ označme jako $h_i$. Relace $\rightarrow_i$ značí kauzální závislost nad procesem $p_i$, tj. zápis $e^x_i \rightarrow_i e^y_i$ značí, že událost $e^x_i$ se stala před událostí $e^y_i$ v~procesu $p_i$. Dále definujme relaci $\rightarrow_{msg}$, která značí kauzální závislost mezi dvojicí událostí $send(m)$ a $recv(m)$.

Nechť $H=\bigcup\limits_{i \, \in\, n} h_i$ značí množinu všech událostí vykonaných v~paralelním programu. Pak \textit{relace kauzálního uspořádání (tj. relace \textbf{Hapens Before})} $\rightarrow$ je definována následovně \cite{cite:PDI}:
$$\forall e_i^x, \forall e_j^y \in H,\: e_i^x \rightarrow e_j^y \Leftrightarrow 
\left\{
	\begin{array}{ll}
       e_i^x \rightarrow e_j^y,\quad pokud \quad i=j \wedge x < y \\
       nebo\\
       e_i^x \rightarrow_{msg} e_j^y \\
       nebo\\
       \exists e_k^z \in H:  e_i^x \rightarrow e_k^z \wedge  e_k^z \rightarrow e_j^y
    \end{array}
\right.$$

Výše definovaná relace \textit{Happens before} umožňuje definovat uspořádání mezi dvěma událostmi. Pokud pro dvě události $e_i$ a $e_j$ platí $e_i \rightarrow e_j$, pak můžeme říct, že událost $e_i$ \textit{předchází} událost $e_j$.
To znamená, že v~systému nemůže nastat situace, kdy by se událost $e_j$ stala před událostí $e_i$. Pokud pro stejné dvě události platí $e_i \centernot\rightarrow e_j$, pak událost $e_i$ \textit{nepředchází} událost $e_j$. Pro výše definované vztahy platí následující pravidla \cite{cite:PDI}:

$$
\begin{array}{lcr}
e_i \centernot\rightarrow e_j & \centernot\implies & e_j \centernot\rightarrow e_j \\
e_i \rightarrow e_j & \implies & e_j \centernot\rightarrow e_j
\end{array}
$$

Posledním možným vztahem dvou událostí je \textit{souběžnost} (concurrency). Dvě události jsou souběžné, pokud platí \cite{cite:PDI}:
$$e_i \centernot\rightarrow e_j \wedge e_i \centernot\rightarrow e_j.$$

Tento vztah se značí jako $e_i \parallel e_j$. Z~hlediska odhalování chyb v~paralelních programech je tento vztah nejdůležitější. Pokud jsou dvě události souběžné (tj. konkurentní), pak nelze říci, která událost proběhne dříve. Může se stát, že při jednom spuštění bude nejprve provedena událost $e_i$ a~poté událost $e_j$, ale při jiném spuštění bude jejich pořadí opačné. Pokud budeme uvažovat reálný příklad, pak je tento problém například v~souběžném zápisu více vláken do jedné sdílené proměnné (viz kapitola \ref{chyby}). 

\section{Logický čas}\label{logic-time}

V~předchozí kapitole bylo definováno uspořádání událostí v~jednotlivých procesech. Každé takové události je přiřazeno časové razítko, které značí čas procesu v~době provádění této události. Nejedná se ovšem o~běžný fyzický čas, ale o~čas \textit{logický} \cite{cite:HB}. Logický čas je dostačující k~určení uspořádání událostí a není nijak závislý na fyzickém čase.

Systém logického času se skládá z~časové domény $T$ a logických hodin $C$, přičemž na množině $T$ je definována relace částečného uspořádání $<$, která není ničím jiným, než výše definovanou relací Happens before. Logické hodiny jsou funkce, která mapuje události $e$ z množiny všech událostí $H$ na prvky z~množiny $T$ ($C : H \rightarrow T$). Pomocí této funkce je možné určit relaci Happens before mezi dvěma událostmi $e_i$ a $e_j$ následovně:
$$e_i \rightarrow e_j \implies C(e_i) < C(e_j)$$

%Colin J. Fidge (February 1988). "Timestamps in Message-Passing Systems That Preserve the Partial Ordering" (PDF). In K. Raymond (Ed.). Proc. of the 11th Australian Computer Science Conference (ACSC'88). pp. 56–66. Retrieved 2009-02-13.
%Mattern, F. (October 1988), "Virtual Time and Global States of Distributed Systems", in Cosnard, M., Proc. Workshop on Parallel and Distributed Algorithms, Chateau de Bonas, France: Elsevier, pp. 215–226

Logický čas může být implementován následujícími způsoby \cite{cite:PDI}:
\begin{itemize}
\item \textit{Skalární čas} (Scalar time) -- proces si uchovává čas jako číslo $d$, které se po každé atomické události inkrementuje. Pokud proces odešle zprávu $m$, pak do zprávy vloží hodnotu čísla $d$. Při přijetí zprávy $m$ proces přečte hodnotu času vloženého do zprávy $d_{msg}$ a svoji hodnotu času $d$ upraví následovně: $d = \max(d, d_{msg})$.

\item \textit{Vektorový čas} (Vector time) -- každý proces uchovává vektor $v$, který má $n$ položek ($n$ je počet procesů). Při provedení interní události proces $p_i$ inkrementuje hodnotu na pozici $v_i$, která označuje čas procesu $p_i$. Pokud proces odesílá zprávu, pak stejně jako u~skalárního času přidává do zprávy právě tento svůj čas. Pokud proces přijme zprávu od procesu $p_j$, pak aktualizuje hodnotu $v_j$, která značí povědomí procesu $p_i$ o~lokálním času procesu $p_j$.

\item \textit{Maticový čas} (Matrix time) -- rozšíření vektorového času, kde každý proces uchovává čtvercovou matici řádu $n$. Matice obsahuje povědomí o~vektorovém čase každého procesu.  
\end{itemize}

% Kapitola synchronizace?
% Synchronizace: zajištění kooperace mezi paralelně (souběžně) prováděnými procesy
%sekce
\section{Synchronizační nástroje}
% Paralelni program je ...
%Prubeh vykonavani je nedeterministicky(planovac, obsluda preruseni, blokujici operace). Synchronizuje se pomoci synchronizacnich primitiv (nizkourovnove na OS, na urovni prog. jazyka - monitor, zamek, atd) 
%Vlakna maji svuj zasobnik, sdileji hladu, problem k pristupu k halde (sdilena pamet) 
Při provádění paralelního programu nastávají situace, kdy je nutné procesy nebo vlákna vzájemně řídit (synchronizovat). Pro řízení souběžného přístupu procesů nebo vláken lze použít následující synchronizační prostředky\footnote{Pokud není uvedeno jinak, pak lze synchronizační prostředky použít jak pro procesy, tak pro vlákna.}:

\begin{itemize}
\item \textit{Semafor} -- může být \textit{binární}, nebo \textit{obecný}. Binární semafor obsahuje metody \texttt{init(value)}, \texttt{lock(value)} a \texttt{unlock()}. Proces čeká při zavolání metody \texttt{lock()}, dokud hodnota semaforu není 0 a~pak jej nastaví na 1. Metoda \texttt{unlock()} nastaví hodnotu semaforu na 0 a~odblokuje procesy čekající na metodě \texttt{lock()} stejného semaforu. Obecný semafor má oproti binárnímu kapacitu, která značí kolik jednotek zdroje chráněného semaforem je k~dispozici. Binární semafor je obecný semafor s~kapacitou 1. V~Javě je obecný semafor implementován třídou \texttt{java.util.concurrent.Semaphore} \cite{cite:OS2,cite:javad}.

\item \textit{Mutex} -- binární semafor určený pro vzájemné vyloučení. Při zamknutí mutexu je uložen jeho vlastník a pouze tento vlastník jej může odemknout. V~Java SE neexistuje třída reprezentující mutex, nicméně lze pro tento účel upravit třídu \texttt{java.util.concurrent.Semaphore}  \cite{cite:OS2,cite:javad}.

\item \textit{Bariéra} -- umožňuje sadě procesů počkat na všechny ostatní. V~Javě je implementována třídou \texttt{java.util.concurrent.CyclicBarrier} \cite{cite:javad}.

\item \textit{Zámek} -- má podobný princip jako semafor, ale jedná se pouze o~sdílenou proměnnou a tudíž může řídit přístup pouze mezi vlákny, nikoli procesy  \cite{cite:OS1}.

\item \textit{Monitor} -- abstraktní datový typ, ve kterém jsou sdílené proměnné dostupné pouze přes operace monitoru (včetně jejich inicializace). Tyto operace monitoru jsou vzájemně vyloučené. Monitor lze implementovat v~Javě například pomocí tříd \texttt{java.util.concurrent.locks.\-Condition} a \texttt{java.util.concurrent.locks.Lock}. Stejně tak lze použít klíčové slovo \texttt{synchronized} na libovolný objekt (\texttt{java.lang.Object}). Tento objekt obsahuje metody \texttt{wait()} (čekání) a \texttt{notify()} (upozornění ostatních, že již nemusí dále čekat) \cite{cite:OS2,cite:javad}.

\end{itemize}

Výše uvedené synchronizační prostředky představují pouze základní sadu synchronizačních nástrojů. Nicméně další takové nástroje jsou odvozeny právě z~těchto nástrojů. Jak bylo zmíněno, synchronizační nástroje slouží k~řízení souběžného přístupu. V~další sekci budou ukázány základní chyby, které mohou nastat, pokud není synchronizace provedena správně, nebo dokonce vůbec.

%toto jako sekci a ty nize jako subsekce
\section{Chyby v~synchronizaci}\label{chyby}

%Chyby ve vicevlaknovych pristupech chyby bezpecnosti a zivosti (data race, atomicity violation, order violation, deadlock, missed signal, livelock, problem uvaznuti, porušení contractu)
Jak již bylo zmíněno v~kapitole \ref{paralel-program}, chyby v~paralelních programech mohou v~jednom běhu nastat, zatímco v~jiném nemusí. Je to dáno proložením běhů jednotlivých vláken nebo procesů. V~následujícím textu budou prezentovány některé ze základních chyb tohoto typu včetně příkladů\footnote{Příklady budou psané v~pseudokódu vycházejícího z~jazyka Java.}, na kterých bude tento problém demonstrován. Příklady budou vysvětlovány nad vlákny, nicméně pro procesy je situace stejná. 

\subsection{Uváznutí (deadlock)}

Prvním chybou, která je zde představena je \textit{uváznutí}. Uváznutí je situace kdy vlákna čekají na stav, který by mohl nastat, pokud by jedno z~těchto vláken mohlo pokračovat. Vlákna jsou takto blokovány navždy \cite{cite:OS1,cite:OS2}.

Uváznutí může nastat v~příkladu \ref{c:deadlock}, pokud 1. vlákno zamkne zámek \texttt{A} na řádku 6 a ve stejnou chvíli 2. vlákno uzamkne zámek \texttt{B} na tomtéž řádku. V~dalším kroku chce 1. vlákno uzamknout zámek \texttt{B}, ale ten je vlastněn 2. vláknem. Současně chce 2. vlákno uzamknout zámek \texttt{A}, ale ten je vlastněn 1. vláknem. Vlákna tedy čekají, než se zámky uvolní, což nemůže nikdy nastat. Pokud ovšem nastane takový průběh, že 1. vlákno získá oba zámky (řádky 6 a 7) a až poté chce 2. vlákno uzamknout zámek \texttt{B} (řádek 6), pak k~uváznutí nedojde, protože vlákno \texttt{A} dokončí práci a poté uvolní oba zámky. 2. vlákno pak může oba zámky získat. V~tomto příkladu byly popsány 2 průběhy. V~jednom k~chybě došlo a ve druhém ne. V~dalších příkladech už budou demonstrovány jen takové proložení, ve kterých k~chybě dojde. 

\begin{lstlisting}[caption=Pseudokód příkladu uváznutí dvou vláken., frame=none, label={c:deadlock}, numbers=left, xleftmargin=1cm]
Lock A = ...;
Lock B = ...;

// 1. vlakno			// 2. vlakno
void run() {			void run() {
  A.lock();			  B.lock();
  B.lock();			  A.lock();
  B.unlock();			  A.lock();
  A.unlock();			  B.lock();
}				}
\end{lstlisting}

\subsection{Blokování}

Další chybou je \textit{blokování}, které nastává, pokud vlákno čeká na stav, který generuje jiné vlákno, a~toto čekání není nutné z~hlediska synchronizace. K~blokování také dochází pokud vlákno čeká na stav, který nemůže nikdy nastat \cite{cite:OS2}.

V~příkladu \ref{c:block} může nastat situace, kdy jedno vlákno získá zámek \texttt{A} a už nikdy jej neuvolní. Druhé vlákno tak bude pořád blokováno.

\begin{lstlisting}[caption=Pseudokód příkladu blokování dvou vláken., frame=none, label={c:block}, numbers=left, xleftmargin=1cm]
Lock A = ...;

// 1. vlakno			// 2. vlakno
void run() {			void run() {
  while(true)			  while(true)
    A.lock();			    A.lock();
}				}
\end{lstlisting}

\subsection{Stárnutí}

\textit{Stárnutí} je podobný problém jako blokování, avšak čekání vlákna není shora omezeno. Vlákno čeká na splnění podmínky, která nemusí být nikdy platná v~okamžiku testování, ale může nastat situace, kdy platná bude. Pak je vlákno uvolněno a může pokračovat. Pokud by k~uvolnění nemohlo nikdy dojít, tak by se jednalo o~blokování \cite{cite:OS2}.

\subsection{Časově závislá chyba nad daty}\label{data-race}

K~\textit{časově závislé chybě nad daty} může dojít, pokud dochází k~souběžnému přístupu více vláken ke sdílené proměnné a alespoň jeden z~přístupů je zápis \cite{cite:OS2}.

V~příkladu \ref{c:race} se nachází sdílená proměnná \texttt{value}, která reprezentuje aktuální hodnotu. Dále se zde nachází dvě vlákna, kde 1. vlákno zapisuje aktuální hodnoty a 2. vlákno vypisuje aktuální hodnoty na výstup. Pokud jsou obě vlákna spuštěny současně, pak může dojít ke dvěma výsledků. Na výstupu se objeví hodnota $5$ nebo $8$ podle toho, zda bude nejdříve aktuální hodnota vypsána, nebo aktualizována.

\begin{lstlisting}[caption=Pseudokód příkladu časově závislé chyby nad daty., label={c:race}, frame=none, numbers=left, xleftmargin=1cm]
int value = 5;

// 1. vlakno			// 2. vlakno
void run() {			void run() {
  value = 8;			  print(value);
}				}
\end{lstlisting}

\subsection{Porušení atomicity}%(citace http://www.jot.fm/issues/issue_2004_06/article5.pdf)
K \textit{porušení atomicity} dojde, pokud vlákno získá stav nějaké sdílené proměnné, další výpočet vlákna závisí na tomto stavu a zárověň během tohoto výpočtu k~této proměnné může přistoupit jiné vlákno a její stav změnit \cite{cite:OS2}.

Následující příklad vychází z~předchozího příkladu. K~porušení atomicity například dojde, pokud 2. vlákno přečte stav proměnné \texttt{value} a uloží si jej do proměnné \texttt{tmp}, následně 1. vlákno aktualizuje hodnotu \texttt{value} na hodnotu 8 a nakonec 1. vlákno provede také aktualizaci této proměnné, ale již na základě neplatného stavu uloženého v~proměnné \texttt{tmp}. Na výstupu se tak může objevit některá z~množiny hodnot: ${6, 8, 9}$.
\begin{lstlisting}[caption=Pseudokód porušení atomicity., frame=none, numbers=left, xleftmargin=1cm]
volatile int value = 5;

// 1. vlakno			// 2. vlakno
void run() {			void run() {
  value = 8;			  int tmp = value;
}				  value = tmp + 1;
				  print(value);
				}			
\end{lstlisting}

\subsection{Porušení kontraktu}

Poslední zde uvedenou chybou je \textit{porušení kontraktu} \cite{cite:contract0}. Kontrakt je sekvence veřejných metod objektu, která musí být vykonána atomicky, s~ohledem na ostatní veřejné metody stejného objektu. Problematika kontraktů je dále rozebírána v~kapitole \ref{contract-basic}.

... vcetne popisu kontraktu ... vice popsat


\section{Metody testování}
V~předchozí kapitole byla vysvětlena problematika paralelních programů a chyb, které se v~nich mohou vyskytovat. V~této kapitole proto budou popsány metody testování, pomocí kterých lze chyby v~paralelních programech odhalit. Největší důraz zde bude kladen na \textit{dynamickou analýzu} a~metodu \textit{vkládání šumu}, neboť tyto dvě metody souvisí s~dalšími částmi této práce.

\subsection{Statická analýza}

\textit{Statická analýza} zkoumá software bez jeho spouštění. Tato metoda je například používána pro hledání syntaktických chyb a je typicky spouštěna před překladem. Nicméně i tuto metodu je možné využít pro testování paralelních programů. Nevýhodou této analýzy je produkování velkého množství \textit{false alarmů}\footnote{\textit{False alarm} je varování, že se může vyskytnout chyba, přestože ve skutečnosti nikdy vzniknout nemůže.}, neboť statická analýza nemá k~dispozici informace o~konkrétních instancích objektů. Existují ovšem metody založené na provádění testovaného softwaru, které se tento problém snaží odstranit. Jednou z~nich je následující metoda \textit{dynamické analýzy} \cite{cite:advances}.

\subsection{Dynamická analýza}\label{dynamic-analysis}

\textit{Dynamická analýza} zkoumá software na základě jeho provádění, během kterého shromažďuje informace o~jeho běhu (události a jejich uspořádání, stavy zámků, vláken, paměťových míst, atd.). Kromě hlášení chyb, které v~daném běhu nastaly se dynamická analýza snaží extrapolovat nasbírané informace a odhalit chyby, které nenastaly. Přesto dynamická analýza nedokáže odhalit všechny chyby, ale pouze ty, které lze odvodit z~běhů, kterých byla svědkem. Z~tohoto důvodů bývá tato analýza spouštěna opakovaně a kombinována například s~metodou \textit{stochastického vkládání šumu} nebo \textit{deterministického testování}. Díky těmto metodám je zvyšován počet testovaných proložení programu, a tím i pravděpodobnost nalezení konkurentních chyb. Samotná analýza zatěžuje systém, což má podobný efekt jako vkládání šumu. Tento jev se označuje jako \textit{noise effect} a je třeba s~ním při analýze počítat. Dynamická analýza je většinou zaměřena pouze na určitý typ chyb a v~takovém případě sbírá pouze informace související s~tímto typem. U~různých typů dynamické analýzy se rozlišují dvě vlastnosti \textit{sound} a \textit{precise}. Pokud dynamická analýza splňuje první vlastnost, pak nemůže přehlédnout chybu. Pokud splňuje druhou vlastnost, pak neprodukuje false alarmy. Nicméně platí, že dynamická analýza nemusí splňovat ani jednu z~nich, tj. může přehlížet chyby a zároveň může produkovat false alarmy. Programy provádějící dynamickou analýzu se nazývají \textit{dynamické analyzátory} (viz kapitola \ref{dynamic-analysators}) a jejich příkladem jsou: \textit{Eraser}, \textit{GoldiLocks}, \textit{FastTrack} nebo \textit{DJIT+} \cite{cite:advances,cite:eraser,cite:ft,cite:rr}.

\subsection{Deterministické testování}

Deterministické testovaní je metoda založená na opakovaném provádění testovaného softwaru, přičemž má plnou kontrolu nad jeho prováděním. K~tomuto účelu je použitý \textit{deterministický plánovač}, který je implementován například pomocí vkládání silného šumu. Cílem je otestovat co nejvíce možných proložení vláken. V~každém kroku provádění softwaru analyzátor zkoumá, jaké možnosti v~plánování mohou nastat a ty ukládá do stavového prostoru testovacích scénářů. Při dalších bězích se provádí další scénáře z~tohoto stavového prostoru, které se opět dále větví. Z~tohoto důvodu jsou zavedeny některé heuristiky, které omezují velikost stavového prostoru. Stavový prostor všech možných scénářů je příliš velký a u~větších programů by testování trvalo příliš dlouho. Často se tak využívá například testování, kdy plánovač nechává běžet vždy pouze jedno vlákno a ostatní nechává pozastavené, nebo je například omezen maximální počet přepnutí kontextu. Případ, kde se testují všechny proložení se nazývá \textit{full model checking} \cite{cite:advances}.

\subsection{Zátěžové testování}

\textit{Zátěžové testování} je založeno na principu vytvoření nejhoršího možného prostředí, ve kterém aplikace může běžet. V~souvislosti s~hledáním konkurentních chyb je to vytvoření velkého množství vláken, které budou navzájem soupeřit o~sdílené zdroje. Tímto přístupem mohou být odhaleny některé chyby, ale s~největší pravděpodobností se bude jednat o~chyby častěji se vyskytující. Vzhledem tomu, že proložení prováděné testovaným softwarem jsou náhodné, může toto testování způsobovat opakované provádění již prozkoumaných proložení, přestože existují jiné proložení, které otestovány nebyly. Další nevýhodou je značné zatížení jak testovaného softwaru, tak prostředí, kde testování běží \cite{cite:advances}.

\subsection{Vkládání šumu}

Tento typ testování vkládá šumu (\textit{noise}) \cite{cite:sum} do prováděného kódu, který opožďuje vlákna a tak může dojít k~proložení, které by jinak nastalo pouze velmi výjimečně. Stejně tak mohou být odhaleny chyby souběžného přístupu dvou instrukcí, které jsou v~kódu dostatečně daleko od sebe a za normálních okolností by nebylo prakticky možné, aby se vykonaly souběžně. Tato chyba může být odhalena vložením dostatečné \textit{silného}\footnote{\textit{Silným} šum pozastaví provádění vlákna na delší dobu.} šumu do jednoho z~vláken. Pomocí metody vkládání šumu může být prozkoumáno velké množství scénářů v~relativně krátké době. Náhodné vkládání šumu nemusí být příliš efektivní, protože může docházet k~vkládání na místa, které nijak neovlivňují proložení vláken a šum by tak pouze zatěžoval systém. Lepších výsledků je dosaženo v~případě použití některé heuristiky, která určuje vkládání šumu pouze na specifická místa v~kódu (\textit{noise seeding problem}) \cite{cite:advances}. Stejně důležitým faktorem, jako je umístění šumu, je i vkládání vhodného typu šumu (\textit{noise seeding problem}). Typ je dán sílou šumu a operací, která šum představuje \cite{cite:advances}.

V~programovacím jazyce Java lze použít například funkce \cite{cite:javad}:
\begin{itemize}
\item \texttt{yield()} -- způsobí přepnutí kontextu (síla udává počet zavolání této funkce, než může vlákno pokračovat),
\item \texttt{sleep()} -- blokuje (uspí) vlákno po zadanou dobu (síla šumu),
\item \texttt{wait()} -- jako \texttt{sleep()} s~rozdílem, že vlákno čeká na objektu \textit{monitoru}.
\end{itemize}

\section{Nástroje pro testování Java programů}
% model checker  Edmund M. Clarke; Orna Grumberg and Doron A. Peled (2000) "Model Checking", MIT Press, ISBN 0-262-03270-8 - pro model checker

Pro testování paralelních programů v~jazyce Java existuje celá řada programů, a proto zde budou představeny především ty, jež souvisejí s~dynamickou analýzou, která je předmětem této práce. Jedním z~nejznámějších je \textit{Java Pathfinder} (JPF) \cite{cite:jpf}, který byl vyvinut v~\textit{NASA Ames Research Center}. JPF dokáže vykonávat testovaný program a ukládat, porovnávat a obnovovat stavy tohoto programu. Díky těmto vlastnostem je používaný jako tzv. \textit{model checker}, ale lze v~něm také nadefinovat dynamické analyzátory. Podobné nástroje jako JPF jsou například \textit{Bandera} nebo \textit{CBMC}, který byl původně vyvinut pro C/C++, ale rozšířen o~podporu jazyka Java.

Dalším nástrojem je projket \textit{IBM ConTest} \cite{cite:IBM} sloužící pro instrumentaci a dynamickou analýzu Java programů. Tento nástroj je velmi podobný dalšímu nástroji \textit{RoadRuner} \cite{cite:rr} (viz kapitola \ref{roadrunner}), který lze rovněž využít pro instrumentaci a dynamickou analýzu. V~neposlední řadě existuje nástroj \textit{Java Race Detector \& Healer}\footnote{Nástroj dostupný na adrese \url{http://www.fit.vutbr.cz/research/groups/verifit/tools/racedetect/}} vyvinutý výzkumnou skupinou VeriFIT, který slouží pro detekci časově závislých chyb. Tento projekt využívá pro instrumentaci programu zmíněný IBM ConTest.

Jak bylo vysvětleno v~kapitole \ref{dynamic-analysis}, tak dynamická analýza nedokáže odhalit všechny chyby během jednoho běhu, a proto je nutné ji spouštět opakovaně. Pro tento účel slouží nástroj \textit{SearchBestie} \cite{cite:sb}, který byl opět vyvinut výzkumnou skupinou VeriFIT. SearchBestie společně s~nástrojem RoadRunner jsou využívány v~další části této práce a proto budou nyní popsány detailněji.

\subsection{RoadRunner}\label{roadrunner}
\textit{RoadRunner} \cite{cite:rr} je framework navržený pro dynamickou analýzu vícevláknových Java programů, který je také napsán v~jazyce Java.
\textit{RoadRunner} vkládá instrumentační kód do bytekódu testovaného programu, což umožňuje testovat programy i bez znalosti zdrojových kódů. Vložený instrumentační kód generuje \textit{tok událostí}, které nastávají v~testovaném programu. Těmito událostmi jsou například:
\begin{itemize}
\item události synchronizace na zámcích,
\item přístup k~proměnným,
\item vytvoření/ukončení vláken,
\item vstup/výstup do/z metod a další.
\end{itemize}

Tok události je zpracováván pomocí \textit{nástrojů}, které mohou filtrovat události a je možné je skládat do tzv. \textit{řetězce nástrojů}. Tímto způsobem lze skládat složitější dynamické analyzátory z~jednodušších kroků, kde každý krok je reprezentován jednodušším nástrojem.

%tool-specific udalostni handlery updatují instrumentační stav kazde operace v pozorovanem behu a muzou reportovat chyby

%\subsection{Nástroje frameworku RoadRunner}
Nástroje jsou tedy stavební kameny všech analyzátorů implementovaných v~tomto frameworku. Při vytváření nástrojů je nutné rozšířit třídu obecného nástroje, tedy třídu \texttt{roadrunner.simple.Tool}. Tato třída definuje metody pro zpracování všech událostí (tzv. \textit{handlery událostí}) z~toku událostí. Každá tato metoda má jako parametr objekt reprezentující zachycenou událost. Pokud určitý nástroj nezpracovává události určitého typu, pak tyto události musí přeposlat dalšímu nástroji v~řetězci nástrojů.
Tabulka \ref{t:rr:tools} obsahuje vybrané nástroje, které již jsou v~tomto frameworku implementovány.

\begin{table}[!htb]
\label{t:rr:tools}
\centering
\label{tab-readrunner}
\begin{tabular}{|l|l|}
\hline
\textbf{Název nástroje} & \textbf{Popis}                                                                            \\ \hline
ThreadLocal             & Filtruje přístup k~lokálním datům.                                                        \\ \hline
ReadOnly                & Filtruje přístup k~datům určeným pouze pro čtení.                                         \\ \hline
ProtectingLock          & Filtruje operace nad zámky, které jsou chráněný dalšími zámky.                            \\ \hline
LockSet                 & Detekuje časově závislé chyby nad daty s~použitím LockSet algoritmu.                      \\ \hline
EraserWithBarrier       & \begin{tabular}[c]{@{}l@{}}Detekuje časově závislé chyby nad daty s~použitím LockSet algoritmu\\ a analýzy bariér.\end{tabular}     \\ \hline
HappensBefore           & \begin{tabular}[c]{@{}l@{}}Detekuje časově závislé chyby nad daty s~použitím algoritmu\\ VectorClock.\end{tabular}                  \\ \hline
DJIT+                   & \begin{tabular}[c]{@{}l@{}}Detekuje časově závislé chyby nad daty s~použitím optimalizovaného\\ algoritmu VectorClock.\end{tabular} \\ \hline
MultiRace               & \begin{tabular}[c]{@{}l@{}}Detekuje časově závislé chyby nad daty s~použitím hybridní\\ LockSet/VectorClock analýzy.\end{tabular}   \\ \hline
Goldilocks              & \begin{tabular}[c]{@{}l@{}}Detekuje časově závislé chyby nad daty s~použitím rozšířeného LockSet\\ algorimu.\end{tabular}           \\ \hline
FastTrack               & Detekuje časově závislé chyby nad daty s~použitím FastTrack algoritmu.                    \\ \hline
\end{tabular}
\caption{Tabulka obsahující vybrané implementované nástroje v~projektu RoadRunner \cite{cite:sb}}
\end{table}

%dopsat tool ktery vklada tool a vraci vysledek search bestie

%RoadRunner dále umožňuje vytvářet a přiřazovat pomocné \textit{stavy} ke každému vláknu, objektu zámku a lokaci v paměti. Do těchto stavů mohou nástroje ukládat data důležitá pro konkrétní analýzu.
Pro každý objekt \textit{Thread}, který využívá JVM pro reprezentaci vláken, vytváří \textit{Roadrunner} objekty typy \textit{ShadowThread}, které jim přiřadí\footnote{Objekt \textit{ShadowThread} obsahuje referenci na originální objekt \textit{Thread}}. Stejným způsobem vytváří také objekty \textit{ShadowLock} pro každý objekt použitý jako zámek a objekt typu \textit{ShadowVar} pro každou lokaci v~paměti. Poslední jmenovaný objekt je uložen v~tzv. \textit{shadow location}, která koresponduje s~každou lokací v~paměti. Zmíněné objekty pak mohou být využity pro uložení specifických informací důležitých pro konkrétní nástroj.
%Tento přístup ovšem vede ke špatně čitelnému kódu a proto se v toolu nachazi Decoration class
%Decorations jsou pouzite jak ohash mapa, ale klicem je samotna trida decoration (kvuli vyssi rychlosti) je pak rychlejsi get a set. value je pak taky trida decoration.
Každý nástroj implementovaný v~\textit{RoadRunneru} by měl splňovat následující podmínky \cite{cite:rr}:
\begin{enumerate}
\item Každý handler události musí vyvolat stejný handler v~dalším nástroji v~řetězci.
\item Pro identifikaci, který nástroj vlastní \textit{shadow location}, musí nástroj obsahovat \textit{shadow objekt} typu $T$ a dále může ukládat do \textit{shadow location} pouze objekty tohoto typu $T$.
\item Metoda \texttt{makeShadowVar}\footnote{Metoda \texttt{makeShadowVar} je zavolána při prvním přístupu do paměti na danou lokaci. Tj. při prvním přístupu k~proměnné.} každého nástroje musí vracet objekt typu $T$.
\item{Pokud je vyvolán \texttt{access} handler, tj. handler přístupu do paměti a \textit{shadow location} je vlastněna tímto nástrojem, tj. \textit{shadow location} obsahuje objekt typu $T$, pak nástroj musí provést jednu z~následujících možností:
	\begin{itemize}
	\item Udržet vlastnictví paměťového místa tím, že uloží/ponechá objekt typu $T$ v~\textit{shadow location}.
	\item{Vzdá se vlastnictví paměťového místa pomocí metody \texttt{advance}, která nahradí \textit{shadow location} objektem typu $T_1$, kde $T_1$ značí typ \textit{shadow} objektu následujícího nástroje. \textit{Shadow location} už nikdy nesmí obsahovat objekt typu $T$.}
	\end{itemize}
}
\item Pokud je vyvolán \texttt{access} handler a paměťové místo není vlastněno tímto nástrojem, pak musí nástroj zavolat \texttt{access} handler následujícího nástroje.
\end{enumerate}

Body 4. a 5. říkají, že \textit{shadow location} každého paměťového místa může modifikovat pouze nástroj, který je jeho aktuálním vlastníkem.

\subsection{SearchBestie}\label{searchbestie}
Platforma \textit{SearchBestie} \cite{cite:sb} je platforma určená pro hledání optimálních testů a jejich spouštění napsaná v~jazyce Java. Využívá techniky prohledávání stavového prostoru, jež tvoří kombinace parametrů, se kterými jsou jednotlivé testy spouštěny. Lze ji mimo jiné použít pro nalezení optimálních testů při testování vícevláknových aplikací, čehož bude využito v této práci. SearchBestie nejprve využívala pro instrumentaci programů nástroj IBMContest, který byl později nahrazen nástrojem RoadRunner (viz \ref{roadrunner}). Jejím úkolem je hledat optimální parametry a s~nimi spouštět testování programu právě prostřednictvím nástroje RoadRunner. SearchBestie se skládá z~následujících modulů: %\cite{•}:%idealni dkumentaci

\begin{itemize}
\item \textit{Manager} -- řídí celý proces a nabízí pomocné funkcionality.
\item \textit{Search} -- modul vybírá kombinaci parametrů a test, který má být v~příštím kroku vykonán.
\item \textit{Executor} -- vykonává testy s~vybranými parametry a sbírá výsledky.
\item \textit{Storage} -- ukládá výsledky testů.
\item \textit{Analysis} -- analyzuje výsledky uložené modulem \textit{Storage}.
\end{itemize}

SearchBestie na vstupu přijímá konfigurační soubor, ve kterém jsou definovány mimo jiné parametry ovlivňující vkládání šumu. Kombinace těchto parametrů vytváří stavový prostor možných konfigurací testů, který SearchBestie postupně prochází. Tyto parametry jsou:

\begin{itemize}
\item \textit{NoiseFreqency} -- udávající sílu šumu.
\item \textit{NoiseStrength} -- udávající frekvenci šumu.
\item \textit{NoiseType} -- udávající typ vkládaného šumu.
\end{itemize}

Na základě jednotlivých konfigurací (tj. stavů) ve stavovém prostoru vkládá RoadRunner šum do testovaného programu a tím ovlivňuje možné proložení vláken.

Ve vstupním souboru lze také definovat další parametry, mezi které patří:

\begin{itemize}
\item \textit{Searcheng} -- definuje, jakým způsobem se bude prohledávat stavový prostor.
\item \textit{Executor} -- v tomto elementu se nachází nastavení pro modul \textit{Executor}. Je zde možné například nastavit maximální počet spuštění testů.
\item \textit{Fitness} -- definuje výpočet fitness funkce.
\item \textit{Javatest} -- tento element definuje jaký typ testu bude spuštěn a nastavení tohoto testu. Vzhledem k použití RoadRunneru se jedná o \textit{RoadRunnerTest} a nastavení všech parametrů, které jsou potřebné pro jeho spuštění. 
\end{itemize}

\subsection{Propojení SearchBestie a frameworku Roadrunner}\label{propojeni-sb-rr}

Jak bylo zmíněno v předchozí kapitole, SearchBestie je nástroj určený pro hledání optimálních parametrů testů a jejich spouštění. Samotné testy jsou tedy prováděny samostatným nástrojem, kterým je v tomto případě RoadRunner. SearchBestie následně sbírá výsledky z těchto testů a na jejich základě vytváří stavový prostor. Pro propojení je důležitá zejména třída \textit{RRPlugin} v RoadRunneru, která musí být použita jako první nástroj v řetězci nástrojů. Tato třída obsahuje tzv. \textit{CoverProucery}, které zaznamenávají informace o chybách, vyskytujících se v testovaném programu. Při ukončení testu CoverProducery zapíší všechny nasbírané informace o chybách do výstupních souborů, které jsou následně načteny v Searchbestie a jejich obsah je dále zpracován. Každý takový CoverProducer může zaznamenávat různé informace a ty musejí být zpracovány různým způsobem. Z toho důvodu musí existovat odpovídající tzv. \textit{ResultItemCreator}, který dokáže tyto výsledky zpracovat.

V diagramu ref{komunikace-obecna} je znázorněna zjednodušená vzájemná komunikace mezi těmito komponentami. SearchBestie nejprve spustí RoadRunner s patřičnými parametry pomocí metody \textit{performTest}. Následně RoadRunner spustí RRPlugin a další nástroje (včetně nástroje \textit{Tool}), zadané v parametrech. RRPlugin pro každý další nástroj vytvoří patřičný CoverProducer. Jakmile dojde k chybě, která byla detekována některým nástrojem (Tool), pak tento nástroj reportuje chybu odpovídajícímu CoverProduceru. Na konci testování jsou ukončovány jednotlivé nástroje a v této chvíli dá RRPlugin pokyn všem CoverProducerům, aby zapsaly všechny nasbírané informace do výstupního souboru. Jakmile SearchBestie dostane informaci o ukončení RoadRunneru, načte a zpracuje informace z výstupních souborů jednotlivých CoverProducerů.

\begin{figure}[!htb]
    \centering
    \includegraphics[width=\textwidth]{obrazky-figures/komunikace-v-systemu-obecna.pdf}
    \caption{Komunikace v systému.}
    \label{komunikace-obecna}
\end{figure}



\chapter{Dynamické analyzátory}\label{dynamic-analysators}

Dynamické analyzátory jsou programy provádějící dynamickou analýzu, přičemž každý z~nich je zaměřen na hledání určitého typu chyb. Typickým problémem je hledání časově závislých chyb nad daty. Analyzátorů, které hledají tento typ chyb je hned celá řada. Nejjednodušším z~nich je algoritmus \textit{Eraser}, který zkoumá program s~pomocí množin zámků držených jednotlivými vlákny, přičemž princip tohoto algoritmu dále rozšiřuje další analyzátor \textit{Goldilocks}. Dalšími analyzátory jsou například \textit{DJIT+} nebo \textit{FastTrack}, které hledají chyby za pomoci vektorového času. Příkladem dynamického analyzátoru, který hledá jiný typ chyby může být analyzátor kontraktů. Právě tento analyzátor společně s~analyzátorem \textit{FastTrack} bude dále více popsán \cite{cite:ft,cite:eraser,cite:contract1}.

\section{FastTrack}\label{fasttrack}

\subsection{Popis analyzátoru}

\textit{FastTrack} \cite{cite:ft} analyzátor patří do skupiny \textit{precizních}\footnote{Dynamický analyzátor je precizní, pokud splňuje vlastnost \textit{precise} (viz kapitola \ref{dynamic-analysis})} analyzátorů a je určen pro odhalování časově závislých chyb nad daty. Využívá adaptivní reprezentaci relace Happens before, díky které dokáže významně zvýšit svou rychlost oproti jiným precizním analyzátorům, jako jsou například BasicVC nebo DJIT+. Běžné precizní dynamické analyzátory ukládají pro každé vlákno vektorový čas posledního zápisu do každé proměnné $x$. Vektorový čas je $n$-tice čísel, kde $n$ je počet vláken (viz kapitola \ref{logic-time}). Všechny operace nad vektorovým časem mají tudíž složitost $O(n)$. Oproti tomu FastTrack může ukládat pouze informaci o~posledním zápisu do každé proměnné $x$ napříč všemi vlákny, a to v~případě, že všechny zápisy do proměnné $x$ jsou uspořádané relací Happens before. Tato informace, nazývaná \textit{epocha}, se skládá z~času posledního zápisu a identifikátoru vlákna, ve kterém zápis proběhl. Veškeré operace nad touto epochou je následně možné provést v~konstantním čase $O(1)$. Stejně jako pro zápis ukládá FastTrack informaci pouze o~posledním čtení každé proměnné $x$, pokud jsou opět všechny předchozí čtení uspořádány pomocí relace Happens before. Pokud nastane případ, kdy některé čtení nebo zápisy nejsou touto relací uspořádány, FastTrack ukládá informaci o~časech vykonání těchto operací pomocí vektorových hodin \cite{cite:ft}.

Protože nutnost ukládat plný vektorový čas je pouze v~minimálním množství situací, dokáže FastTrack díky použití epoch snížit režii spojenou s~analýzou z~$O(n)$ až na $O(1)$. Jak bylo popsáno výše, FastTrack analyzátor ukládá informace o~posledním zápisu pomocí epochy. Vlákna ovšem pro reprezentaci logického času používají vektorový čas a proto je nutné tyto dvě formy umět porovnat. \textit{Epocha} je definována jako dvojice $c@t$, kde $c$ je logický čas vlákna $t$ a platí, že tato epocha předchází vektorový čas $V$ (označováno jako $c@t \preceq V$) v~relaci Happens before, jestliže platí $c\leq V(t)$ \cite{cite:ft}. 

Analyzátor rozlišuje následující 3 typy situací, které mohou nastat při přístupu ke sdílené proměnné \cite{cite:ft}:
\begin{itemize}
\item čtení s~následným zápisem,
\item zápis s~následným čtením,
\item zápis následující dalším zápisem.
\end{itemize}

\subsection{Zápis následovaný čtením nebo zápisem}

Detekování dvou konkurentních zápisů je pomocí porovnání epochy s~vektorovým časem jednoduché. Pokud označíme epochu posledního zápisu do proměnné $x$ jako $W_x$, pak ve chvíli následujícího zápisu do této proměnné v~čase $V$ stačí porovnat tento vektorový čas s~epochou $W_x$. Stejným způsobem lze také rozhodnout, zda mohlo dojít k~časově závislé chybě nad daty při zápisu následovaném čtením. Ve chvílí čtení proměnné $x$ v~čase $V$, se opět porovná tento vektorový čas s~epochou posledního zápisu $W_x$, tedy zda platí: $W_x \preceq V$. Pokud tato rovnost platí, pak nedošlo k~časově závislé chybě nad daty \cite{cite:ft}.

\subsection{Čtení následované zápisem}

Detekování možného souběžného přístupu při čtení s~následované zápisem není v~analyzátoru FastTrack tak jednoduché, jako předchozí dva případy. Důvodem je to, že jednotlivá čtení libovolné proměnné $x$ nemusejí být totálně uspořádána, jako tomu je u~zápisu\footnote{Pokud dojde k~porušení totálního uspořádání u~zápisu, pak byla nalezena časově závislá chyba nad daty.}. Pokud nastane případ, že některá čtení nejsou totálně uspořádána, pak je nutné uchovat celý vektorový čas tohoto čtení. Tento případ může nastat pouze v~případě, že je proměnná $x$ sdílena a čtení není chráněno zámkem. V~ostatních případech postačí uchování epochy. To zda došlo k~souběžnému přístupu lze opět určit podobně jako v~předchozích případech, ale může nastat stav, kdy budou porovnávány dva vektorové časy, konkrétně $R_x$ (čas posledního čtení proměnné $x$) a aktuální čas zápisu $V$. V~tomto případě bude porovnání časově nejnáročnější, nicméně k~této situaci dochází pouze velmi zřídka \cite{cite:ft}.

\section{Analyzátor kontraktů}\label{contract-all}

\subsection{Definice základního kontraktu}\label{contract-basic}

\textit{Kontrakt} \cite{cite:contract0} byl původně definován jako sekvence příkazů s~definovanými podmínkami. Pokud je sekvence vykonána bez splnění těchto podmínek, dochází k~porušení tohoto kontraktu. \textit{Kontrakt pro souběžnost} \cite{cite:contract01} je oproti tomu protokol přístupu veřejných služeb modulu (tj. veřejných metod). Každý modul může mít definován vlastní kontrakt, který obsahuje množinu sekvencí služeb (metod). Podmínkou splnění kontraktu je atomické vykonání těchto sekvencí, pokud jsou vykonávány nad stejným objektem.

Pro další popis je nutné zavést následující značení, které vychází z~článků \cite{cite:contract1} a \cite{cite:contract2}. Nechť je množina jmen veřejných metod modulu označena jako $\Sigma_M$, dále kontrakt jako množina klauzulí $R$, kde každá klauzule $\varrho \in R$ je regulární výraz nad $\Sigma_M$. K~porušení kontraktu dojde, pokud sekvence reprezentovaná kontraktem není provedena atomicky nad stejným objektem $o$, tj. sekvence kontraktu je proložena alespoň jednou metodou z~množiny $\Sigma_M$.

Příkladem kontraktu (označme jej $\varrho_1$) může být sekvence metod \texttt{indefOf} a \texttt{get} nad objektem typu \textit{seznam}. Pokud by tato sekvence byla proložena například metodou \texttt{add}, která by vložila prvek na první místo seznamu, pak by index hledaného prvku vrácený metodou \texttt{indexOf} již nebyl platný a následné použití metody \texttt{get} by vrátilo jiný prvek seznamu. Tento kontrakt lze zapsat jako: $$\varrho_1: \texttt{indefOf} \quad \texttt{get}.$$


%toto presunout na konec dalsi subsekce?
%Porušení kontraktu lze odhalit pomocí statické nebo dynamické analýzy. %\textit{Statická analýza} se snaží formálně dokázat, že nemůže dojít k žádnému porušení kontraktu. Statická analýza tímto způsobem může odhalit všechny porušení kontraktu, nicméně na druhé straně může produkovat false alarmy, protože nemá přesné informace o konkretních objektech. Oproti tomu \textit{Dynamická analýza} neprodukuje false alarmy, protože má přesné informace o objektech, nad kterými dochází k volání metod z kontraktů. Na druhou stranu dynamická analýza nemusí odhalit všechny porušení kontraktu a to ty, které se nestaly v daném proložení a ani je nelze odvodit z tohoto proložení a dalších informací nashromážděných během běhu programu. Pokud se ovšem dynamická analýza použije společně s metodou \textit{vkládání šumu}, pak se množina neodhalených kontraktů minimalizuje.

%%%%%% vlozeno %%%%%
\subsection{Rozšíření kontraktu}\label{contract-def-ext}

Rozšíření publikované v~\cite{cite:contract2} vychází z~předchozí definice kontraktů, přičemž spojuje kontrolu atomicity sekvencí kontraktů s~definicí metod, které tuto sekvenci nesmí porušit. Znamená to tedy, že sekvence metod musí být atomická pouze s~ohledem na proložení pouze určenou množinu metod. Dalším rozšířením je přidáním parametrů do jednotlivých metod. 

Uvažujme nejprve \textit{rozšíření kontraktů o~parametry}. Pokud se znovu zaměříme na příklad kontraktu $\varrho_1$, pak je viditelné, že porušení kontraktu způsobí chybu pouze v~případě, kdy je návratová hodnota metody \texttt{indefOf} použita jako parametr metody \texttt{get}. Tento kontrakt s~rozšířením o~parametry lze zapsat jako:
$$\varrho_1^\prime: X = \texttt{indefOf(\_)} \quad \texttt{get(X)}.$$

Dále uvažujme \textit{rozšíření o~kontextové informace} (tj. rozšíření o~metody, které nesmí sekvenci kontraktu porušit). Uvažujme opět příklad kontraktu $\varrho_1$ a základní definicí kontraktu. Tato definice říká, že sekvence kontraktu nesmí být proložena žádnou metodou z~množiny veřejných metod modulu $\Sigma_M$. Uvažujme tedy například metody \texttt{indexOf} a \texttt{remove} z~množiny $\Sigma_M$. Pokud bude sekvence kontraktu $\varrho_1$ proložena voláním metody \texttt{indexOf}, pak k~žádné chybě nedojde. Pokud ovšem bude proložena voláním metody \texttt{remove}, pak k~chybě může dojít. Toto rozšíření tedy umožňuje definovat množinu sekvencí metod (sekvence těchto metod bude dále označována jako \textit{spoiler}), vůči kterým musí být sekvence kontraktu (dále označovaná jako \textit{target}) provedena atomicky \cite{cite:contract2}.

Nechť je tedy $\mathbb{R}$ množina všech \textit{targetů}, kde každý target $\varrho \in \mathbb{R}$ je regulární výraz nad množinou $\Sigma_M$. Nechť $\mathbb{S}$ je množina \textit{spoilerů}, kde každý spoiler $\sigma \in \mathbb{S}$ je regulární výraz nad $\Sigma_M$. Dále označme abecedu všech targetů jako $\Sigma_R \subseteq \Sigma_M$ a všech spoilerů jako $\Sigma_S \subseteq \Sigma_M$. Pak \textit{kontrakt} je definován jako relace $\mathbb{C} \subseteq \mathbb{R} \times \mathbb{S}$, kde pro každý target je definována množina spoilerů, které mohou vyvolat porušení atomicity.

K~porušení kontraktu dojde, pokud je nějaká sekvence prováděných metod $r$, odpovídající targetu $\varrho \in \mathbb{R}$ vykonána nad objektem $o$, plně proložena celou sekvencí metod $s$, odpovídající nějakému spoileru $\sigma \in \mathbb{C}(\varrho)$, která je vykonána nad stejným objektem $o$. Sekvence targetu $r$ je plně proložena sekvencí spoileru $s$ pokud začátek vykonávání $r$ začíná před začátkem vykonávání $s$~a~současně konec vykonávání $s$ předchází konec vykonávání $r$. \cite{cite:contract2}

V~další části této práce budou uvažovány obě tyto rozšíření, přičemž značení kontraktu bude například pro výše zmiňovaný příklad následující:
$$\varrho_1^{\prime\prime}: X = \texttt{indefOf(\_)} \quad \texttt{get(X)} \leftsquigarrow \texttt{remove(\_)}$$.

Kontrakty lze detekovat jak na základě statické analýzy, tak na základě dynamické analýzy. Vzhledem k~tématu této práce zde bude popsána pouze dynamická analýza, která byla navržena jak pomocí LockSet algoritmu, tak pomocí relace Happens before. Tato dynamická analýza byla také implementována v~projektu ANaConDA, která zkoumá paralelní programy psané v~jazyce C/C++.


\subsection{Dynamická analýza založena na LockSet algoritmu}

Prvním možným způsobem detekce kontraktů pomocí dynamické analýzy je analýza založená na množině zámků držených jednotlivými vlákny (\textit{LockSet-based}), jejíž typickým představitelem je algoritmus Eraser. Díky tomuto typu dynamické analýzy je možné rozšířit proložení, které se událo, a tím nalézt porušení kontraktu, které se přímo nestaly. Tento způsob byl představen v~\cite{cite:contract1} a dokáže pracovat pouze ze základní definicí kontraktu.

V~této analýze je nejdříve nutné detekovat kontrakty, které se vyskytují ve vykonávaném programu. Tohoto je dosaženo pomocí konečných automatů, kde pro každou sekvenci každého kontraktu je vytvořen odpovídající konečný automat. Každé vlákno obsahuje množinu instancí těchto automatů, kde instance odpovídá aktuálním nedokončeným sekvencím kontraktů, které se nachází v~aktuálním stavu programu. Pokud je zavolána metoda $m \in \varrho_M$, pak je nad každou instancí automatu proveden pokus o~postup do dalšího stavu pomocí této metody $m$ (tento krok je označován jako \textit{advance}). Pokud je nový stav automatu stavem konečným, pak je detekován výskyt sekvence kontraktu. Pokud je metoda $m$ metodou, kterou začíná některá sekvence kontraktu, pak je vytvořena nová instance kontraktu \cite{cite:contract1}.

Pokud je sekvence kontraktu detekována, je nutné ověřit, zda byla prováděna atomicky. Pomocí informací o~zámcích, držených jednotlivými vlákny během provádění kontraktu, lze detekovat, zda byla celá sekvence chráněna alespoň jedním zámkem. Pokud byla chráněna alespoň jedním zámkem, pak je pravděpodobné, že byla provedena atomicky. Pravděpodobné to je proto, že může nastat případ, kdy tomu tak nebude. Konkretně se jedná o~situaci, kdy dojde k~proložení dvou sekvencí kontraktu nad stejným objektem ve dvou různých vláknech a obě tyto sekvence budou chráněny zámkem, ale tento zámek bude odlišný. Pokud nastane tento případ, mohou být sekvence prováděny souběžně, a tím dojde k~porušení kontraktu. Tato situace je diskutována v~článku \cite{cite:contract1} a jako řešení je navržena dynamická analýza založena na relaci Happens before.


\subsection{Dynamická analýza založena na relaci Happens before}\label{contract-hb}

Dynamická analýza založena na relaci Happens before, představena v~článku \cite{cite:contract2}, je navržena pro rozšířenou definici kontraktu s~kontextovými informacemi. Tato analýza využívá Happens before relaci (viz \ref{happens-before}), kde jako komunikace mezi vlákny jsou uvažovány operace \texttt{acquire} a \texttt{release} nad stejnými zámky a operace \texttt{fork} a \texttt{join}. Tato analýza umožňuje detekci porušení kontraktů za běhu díky technice nazvané \textit{trace window}. Tento koncept umožňuje neuchovávat celou sekvenci všech provedených metod, nýbrž pouze podmnožiny této sekvence (dále označováno jako \textit{okno běhu}), která se postupně pohybuje v~průběhu vykonávání programu. Cílem tohoto konceptu je uchovávat okno o~co nejmenším počtu naposled vykonaných metod.
% Definuje 3 podmínky kdy se jedná o instanci targetu
Události se přidávají do okna běhu $v$~v~momentě, kdy se vyskytnou, a odebírají, pokud již nejsou potřeba. Odebírání je definováno tak, že všechny události z~určité instance targetu (spoileru) se mohou odebrat, pokud již nepatří do žádné jiné aktuálně sledované instance targetu nebo spoileru. Velikost okna tak závisí pouze na počtu instancí targetů a~spoilerů, nikoli na délce běhu programu. Aby bylo možné odstraňovat z~tohoto okna události, je nutné odstraňovat nepotřebné instance targetů a spoilerů. Pro odstraňování těchto instancí jsou definovány následující pravidla \cite{cite:contract2}: 

\begin{itemize}
\item Odstranění instance spoileru $s$ je bezpečné, pokud porušení kontraktu, které může být odhaleno pomocí instance $s$, lze také odhalit bez této instance $s$.
%Toto pravidlo je formálně popsáno \textit{Lemmou 1} v citovanem dokumentu
\item Pro každé vlákno a každý spoiler je nutné uchovávat pouze poslední instanci spoileru. %Lemma 2
\item Pokud existují dvě instance $r_1$, $r_2$ targetu $\varrho \in \mathbb{R}$ takové, že $end(r_1) \rightarrow_{hb} start(r_2)$, pak je bezpečné odstranit instanci $r_1$, pokud $s$ začíná až za \textit{oknem běhu} nebo pokud $start(s) \rightarrow_{hb} start(r_1)$.%Lemma 3
\item Pokud existují dvě instance $r_1$, $r_2$ targetu $\varrho \in \mathbb{R}$ takové, že $end(r_1) \rightarrow_{hb} start(r_2)$. Nechť $s$ je instance spoileru $\sigma \in \mathbb{S}$, kde $(\varrho, \sigma) \in \mathbb{C}$, pak je bezpečné odstranit instanci $r_1$ s~ohledem na $s$~pokud platí že $start(s) \in v \wedge end(s) \notin v$ a zároveň $start(s) \centernot\rightarrow_{hb} start(r_2)$. %Lemma 4
\item Obecně platí, pro každý target $\varrho \in \mathbb{R}$ je uchováno $|T| + 1$ instancí, kde $T$ značí množinu běžících vláken v~\textit{okně běhu}. Pro každé vlákno musí být uchována jedna instance a také 1 navíc pro vlákno, které může být potencionálně vytvořeno.
\end{itemize}


%\subsection{Race Detector & Healer}
%Strucne popsat, ale pouziva IBM contest, takze se neda pouzit?


\chapter{Návrh analyzátoru kontraktů}\label{navrh}

[Poznamky z konzultace] Toto musi byt uplna kapitola, bude nejvetsi. Tady popisu kompletni navrh tak, aby to podle toho slo implementovat
navrh celeho reseni - kazdou cast navrhnu tak jak by se to melo resit
musi to byt uplne tak aby to nekdo mohl vzit a podle toho to mohl naimplementovat. Narazel by na nejake problemy a o tom bude ta dalsi kapitola.
do navrhu same diagramy jak architekturalni, tak behavioralni (kdo s kym jak jedna - state chart nebo sequence diagram (je jedno kterym))

V této kapitole bude popsán návrh dynamického analyzátoru kontraků, který odpovídá popisu ... . Jedná se tedy o analýzu kontraktů s parametry, přičemž pro detekci většího počtu chyb byl využitý princip vektorových hodin.

\section{Specifikace požadavků}\label{spec}

Samotný analyzátor je navržen jako nástroj ve frameworku RoadRunner (viz. ...následující kapitola...), nicméně musí být také upravena platforma SearchBestie, aby tento analyzátor byla schopná spustit a zpracovat jeho výsledky. Z tohoto důvodu budou v následujících požadavcích také požadavky spojené s touto integrací.

\begin{enumerate}\itemsep0em
%\item{Uživatelské požadavky}
\item{Obecné požadavky}
\begin{enumerate}
	\item{Analyzátor musí být implementován projektu RoadRunner.}
	\item{Roadrunner s analyzátorem musí být spustitelný z příkazové řádky.}
	\item{Roadrunner s analyzátorem musí být spustitelný ze SearchBestie.}
	\item{RoadRunneru musí uložit informace o nalezených chybách do výstupního souboru.}
	\item{SearchBestie musí zpracovat tyto výstupní informace.}
	\item{SearchBestie musí na základě nalezených chyb prohledávat stavový prostor parametrů testů.}
\end{enumerate}

\item{Funkční požadavky}
\begin{enumerate}
	\item{Analyzátor získá definici kontraktů z konfiguračního souboru.}
	\item{Musí být definován formát konfiguračního souboru.}
	\item{Cesta ke konfiguračnímu souboru bude předána analyzátoru jako parametr v příkazové řádce.}
	\item{Cesta ke konfiguračnímu souboru bude předána SearchBestie ve vstupním souboru, která ji předá RoadRunneru.}
	\item{Analyzátor musí být schopný zpracovat konfigurační soubor a uložit si kontrakty do vnitřní reprezentace.}

	\item{Porušení kontraktu musí být detekováno v následujících případech:}
		\begin{enumerate}
			\item{Analyzátor musí být schopný detekovat porušení kontraktu, ke kterým skutečně došlo.}
			\item{Analyzátor musí být schopný detekovat porušení kontraktu, ke kterým skutečně nedošlo, ale lze pomocí vektorových hodin odvodit, že k nim může dojít.}
			\item{Analyzátor nesmí detekovat porušení kontraktu v žádném jiném případě (pokud k němu nedošlo).}
		\end{enumerate}	

	\item{Analyzátor musí podporovat kontrakty různých typů:
		\begin{enumerate}
			\item{K jednomu targetu může být zadán jeden spoiler.}
			\item{K jednomu targetu může být zadáno více spoilerů.}
			\item{Targetů může být zadáno v konfiguračním souboru více.}
		\end{enumerate}
	}
	\item{Metody v targetech a spoilerech mohou být:
		\begin{enumerate}
			\item{Bez parametrů,}
			\item{s parametry,}
			\item{s návratovým parametrem.}
		\end{enumerate}
	}
	\item{Metody musí být jednoznačně identifikovány pomocí fully qualified name..}
	\item{Stejně označený parametr v sekvenci metod je považován za shodný parametr.}
	\item{Stejně označený parametr v targetu a jemu připojeném spoileru musí být považován za shodný parametr.}
	\item{Parametr zadaný znakem \texttt{\_} musí být ignorován.}
\end{enumerate}

\end{enumerate}

[Poznamky z konzultace] S: co presne je mojim cilem udelat, kazda odrazka rika co to ma umet, ma to byt spustitelne v prikazove radce, contract - co to ma umet
identifikovane odrazky (cislama, viceurovne - ne vic nez tri urovne) - kazda odrazka bude definovat co to ma umet. I z uzivatelskeho hlediska - napriklad ze to ma byt spustitelne z prikazove radky, ma to umet kontrakty (v odrazkach jak presne), dal ma to umet pracovat se search bestie.


\section{Přehled částí systému}\label{prehled-casti-systemu}

%... TODO diagram ... %stejny jako v propojeni-sb-rr

Jak bylo zmíněno v kapitole, pro nový analyzátor je nutné vytvořit několik nových komponent. Konkrétně se jedná o samotný analyzátor (ContractTool) a jemu přidružený CoverProducer (konkrétně ContractCoverProducer), přičemž obě tyto části se budou nacházet v projektu RoadRunner. Dále je nutné vytvořit podporu tohoto nástroje v SearchBestie, kde se o zpracování nasbíraných výsledků bude starat ResultItemCreator (konkrétně ContractRICreator). Na diagramu \ref{komunikace-contract} je znázorněna komunikace mezi těmito komponentami, přičemž si lze všimnout, že tato komunikace odpovídá standardní komunikaci mezi SearchBestie a Roadrunnerem prezentované v kapitole \ref{propojeni-sb-rr}

\begin{figure}[!htb]
    \centering
    \includegraphics[width=\textwidth]{obrazky-figures/komunikace-v-systemu-contract.pdf}
    \caption{Konkrétní komunikace mezi komponentami dynamického analyzátoru kontraktů a Searchbestie.}
    \label{komunikace-contract}
\end{figure}

\section{Nástroj v Roadrunneru}\label{navrh-rr-nastroj}


Nástroj ContractTool (dále jen CT) ve Frameworku Roadrunner je hlavní částí analýzy kontraktů, neboť se samotná analýza provádí právě zde. CT rozšiřuje třídu Tool frameworku Roadrunner, díky čemuž může zachytávat potřebné události. CT využívá události: init, exit, makeShadowVar, acquire, release, create, preStart a postJoin. První metoda (tj. init) je použita pro inicializaci nástroje, další dvě jmenované metody, exit a makeShadowVar, jsou důležité pro detekci porušení kontraktu (viz ...) a ostatní metody slouží pro práci s vektorovým časem (viz ...). CT dále definuje parametr \texttt{contractFile}, ve kterém musí být zadána cesta ke konfiguračnímu souboru kontraktů. Takto předaný konfigurační soubor je zpracován právě v metodě init za pomoci \textit{Parseru}.

\subsubsection{Konfigurační soubor}\label{config-file}

Parser slouží pro zpracování obsahu konfiguračního souboru, ve kterém se nachází definice kontraktu, a vytvoření ekvivalentní vnitřní reprezentace. Formát konfiguračního souboru vychází z notace použité v definici kontraktů (viz \ref{contract-all}) a lze jej popsat gramatikou \ref{gramatika}. Symbol \texttt{<method_name>} zastupuje tzv. \textit{fully qualified name}, které v Javě běžně využíváno pro jednoznačnou identifikaci. V gramatice se rovněž vyskytuje symbol \texttt{|}, který je symbolem terminálním, nikoli znakem oddělujícím dvě alternativní přepisovací pravidla. Dále gramatika definuje speciální parametr \texttt{\_}, který je použit pro ignorovaný parametr (viz \ref{spec}).

\begin{lstlisting}[escapeinside={(*}{*)}, caption={Gramatika generujícíc validní konfigurační soubor.}, label={gramatika}]
<contract> (*$\rightarrow$*) <target> <- { <spoilers> } <contract>
<contract> (*$\rightarrow$*) (*$\epsilon$*)
<spoilers> (*$\rightarrow$*) <spoiler> <spoilers'>
<spoilers> (*$\rightarrow$*) <spoiler>
<spoilers'> (*$\rightarrow$*) | <spoiler> <spoilers'>
<spoilers'> (*$\rightarrow$*) (*$\epsilon$*)
<target> (*$\rightarrow$*) <method sequence>
<spoiler> (*$\rightarrow$*) <method sequence>
<method sequence> (*$\rightarrow$*) <method> <method sequence>
<method sequence> (*$\rightarrow$*) (*$\epsilon$*)
<method> (*$\rightarrow$*) <ret parameter><method_name>(<parameters>)
<parameters> (*$\rightarrow$*) <parameter><parameters'>
<parameters> (*$\rightarrow$*) <parameter>
<parameters> (*$\rightarrow$*) (*$\epsilon$*)
<parameters'> (*$\rightarrow$*) ,<parameter><parameters'>
<parameters'> (*$\rightarrow$*) (*$\epsilon$*)
<ret parameter> (*$\rightarrow$*) <parameter>:
<ret parameter> (*$\rightarrow$*) (*$\epsilon$*)
<parameter> (*$\rightarrow$*) a-zA-Z0-9
<parameter> (*$\rightarrow$*) _
<method_name> (*$\rightarrow$*) a-zA-Z0-9.$
\end{lstlisting}

Příklad \ref{config-file} ukazuje obsahu souboru \texttt{test1}, který může být vygenerován výše zmíněnou gramatikou a je jedním ze souborů využitých při testování. V tomto souboru je definován kontrakt, kde targetem je sekvence dvou metod (m1 a m2) a spoilerem je jediná metoda (m1).

\begin{lstlisting}[caption={Příklad konfiguračního souboru.},label={config-file}]
cz.vutbr.fit.Test1$Subject.m1() cz.vutbr.fit.Test1$Subject.m2() <-
{ cz.vutbr.fit.Test1$Subject.m1() }
\end{lstlisting}

Takto definovaný konfigurační soubor je zpracován Parserem do vnitřní reprezentace, která je popsána v následující kapitole.

Pro předání cesty ke konfiguračnímu souboru do nástroje Roadrunner je nutné definovat nový parametr \texttt{-contractFile}, který je přidán jako tzv. \textit{Command line option} do nástroje ContractTool. Tento parametr je při spuštění nástroje ContractTool vždy vyžadován.

\subsection{Vnitřní reprezentace}\label{vnitrni-rep}

Analyzátor si musí uchovat definici kontraktu, která obsahuje definici targetů a jim asociovaných spoilerů. Tyto targety a spoilery jsou používány jako šablony, ze kterých se vytvářejí jejich instance. Targety a spoilery tedy obsahují neměnící se informace, kdežto jejich instance svůj obsah mění. Vztah těchto částí je znázorněn na diagramu \ref{class-diagram}

\begin{sidewaysfigure}[!htbp]
    \includegraphics[width=\textwidth]{obrazky-figures/class-diagram.pdf}
    \caption{Diagram zobrazující třídy Target, Spoiler, Instance a jim přidružené.}
    \label{class-diagram}
\end{sidewaysfigure}

Definici kontraktu reprezentuje třída Contract, která se v systému nachází vždy pouze jednou. Tato třída obsahuje instance tříd Target a Spoiler, které dědí od střídy MethodSequence. MethodSequence obsahuje sekvenci instancí třídy MethodTemplate, které reprezentují šablonu metody ze zadání. Tato třída obsahuje důležité informace o metodě, která má být dynamickým analyzátorem sledována. Jedná se o jméno metody, konkrétně \textit{fully qualified name}, jména parametrů a jméno návratového parametru\footnote{Jména parametrů nejsou skutečnými jmény parametrů, nýbrž jmény zástupnými, které slouží pro vyjádření vztahů, mezi více metodami. Více v kapitole ...reakce na prichod exit metody TODO ...}. Poslední nezmíněnou třídou je třída Instance, které reprezentuje instanci aktuálně rozpracovaného, nebo dokončeného, targetu nebo spoileru. Tato instance se mění při detekování vhodné metody, který je provedena v testovaném programu (viz kapitola ... reakce na příchod metody TODO...). Třída Instance obsahuje vektorový čas první a poslední metody v sekvenci metod, odkaz na šablonu této sekvence metod (tj. target nebo spoiler) a jména parametrů ze sekvence metod, společně s již přiřazenými hodnotami. Dále instance obsahuje atribut \textit{actualPos}, tj. index metody ze šablony, která byla jako poslední přijata\footnote{Atribut \textit{actualPos} ukazuje na poslední zpozorovanou událost ze sekvence metod šablony. Tj. pokud je instance vytvořena, je hodnota $actualPos = 0$ (ukazuje na první metodu), a pokud je instance dokončena, pak je hodnota $actualPos = length-1$, kde $length$ je počet metod v sekvenci dané šablony.}. Vztah mezi jmény parametrů a jejich hodnotami bude vysvětlen v následujících kapitolách, ovšem zde je důležité si uvědomit, že jména parametrů se nachází v šabloně metod (MethodTemplate), zatímco jejich hodnoty až v konkrétních instancích.

Počet šablon (tj. tagetů a spoilerů) bude neměnný, ovšem počet jejich instancí bude v průběhu analýzy narůstat. Uložení těchto instancí je tedy navrženo tak, aby se mezi nimi dalo co nejrychleji vyhledávat, tj. instance, které spolu souvisejí, jsou uloženy vždy pohromadě a jsou odděleny od ostatních, které s nimi nesouvisí. K tomuto účelu je využita třída \textit{ShadowVar}, která umožňuje ukládat informace ke každému paměťovému místu (tj. ke každému objektu nebo proměnné). Konkrétně je použita instance třídy \textit{ObjectWindows}\footnote{Třída ObjectWindows dědí od třídy ShadowVar.}, která obsahuje přiřazení objektů \textit{Window} ke každému existujícímu vláknu. Instance této třídy je vytvořena při prvnímu přístupu k paměťovému místu, což je zajištěno v metodě \textit{makeShadowVar}.  Třída \textit{Window} je již třídou obsahující jednotlivé instance targetů a spoilerů. Tento vztah je reprezentován na diagramu \ref{class-object-windows}, kde jsou pro jednoduchost vynechány položky třídy Instance, která je plně zobrazena na diagramu \ref{class-diagram}.

\begin{figure}[!htbp]
    \includegraphics[width=\textwidth]{obrazky-figures/class-diagram-ow.pdf}
    \caption{Diagram zobrazující třídu ObjectWindows a jí přidružené trídy.}
    \label{class-object-windows}
\end{figure}

\subsection{Práce s šablonami a instancemi}\label{sablony-instance}

V dynamickém analyzátoru se tedy vyskytují šablony targetů, spoilerů a jejich instance. V této kapitole bude popsáno, jak spolu souvisí a jak je s nimi pracováno.

Při inicializaci samotného nástroje jsou vytvořeny všechny targety a spoilery z konfiguračního souboru. Následně je zahájena samotná dynamická analýza, která sleduje důležité události. V kontextu práce s šablonami a instancemi je důležitá pouze událost \textit{exit}, která informuje o ukončení vykonání metody $m_{event}$. Tato událost obsahuje informace o objektu, na kterém byla vykonána, identifikaci (tj. jméno metody), hodnoty parametrů, se kterými byla zavolána, a návratovou hodnotu této metody. Důvodem, proč je nutné použít metodu \textit{exit}, namísto metody \textit{enter}, je ten, že při zavolání metody \textit{enter} jsou k dispozici pouze hodnoty vstupních parametrů, ale nikoli návratová hodnota. Oproti tomu v metodě \textit{exit} tato návratová hodnota již známá je, a proto se při dynamické analýze kontraktů s parametry zachytávají vykonané metody až po jejich provedení - tedy v metodě \textit{exit}. Jakmile dojde k detekci této události, pak dynamický analyzátor provede následujících kroky:

\begin{enumerate}
\item{V případě, že metoda $m_{event}$ odpovídá první metodě $m_{first}$ některého targetu, nebo spoileru, pak vytvoří jeho novou instanci.}
\item{V případě, že metoda $m_{event}$ odpovídá metodě $m_{expect}$, která je právě očekávána v některé instanci targetu nebo spoileru, pak provede krok \textit{advance} nad touto instancí.}
\item{Pokud nedojde ani k jednomu z předchozích případů, pak je událost ignorována.}
\end{enumerate}

\subsection{Vytvoření nové instance}\label{new-instance}

Pokud dojde k 1. kroku z předchozí kapitoly \ref{sablony-instance}, je vytvořena nová instance targetu nebo spoileru. V tomto kroku je také zmíněno, že metoda $m_{event}$ musí \textit{odpovídat} první metodě $m_{first}$ některého targetu nebo spoileru. V tomto případě tedy musí dojít k porovnání těchto metod, přičemž při vytváření nové instance ještě nejsou dostupné žádné informace o hodnotách parametrů, a tak je dostačující pouze následující porovnání:

\begin{enumerate}
\item{Metoda $m_{event}$ musí mít stejné jméno\footnote{K porovnávání dochází nad \textit{fully qualified} jménem metod} jako metoda $m_{first}$.}
\item{Metoda $m_{event}$ musí mít stejný počet parametrů jako $m_{first}$.}
\end{enumerate}

Na první pohled by se mohlo zdát, že nedochází ke kontrole typů parametrů, a také výstupního parametru. Tato kontrola ovšem není možná, neboť v definici kontraktů se nacházejí pouze zástupná jména parametrů, ale nikoli jejich typy.

Při vytvoření nové instance targetu nebo spoileru je utné provést následující kroky:

\begin{itemize}
\item{Přiřadit instanci odkaz na šablonu (target nebo spoiler), ze které je vytvářena.}
\item{Aktuální index v sekvenci metod ($actualPos$) nastavit na hodnotu 0.}
\item{Přiřadit instanci identifikátor vlákna, ve kterém byla metoda $m_{event}$ vyvolána.}
\item{Přiřadit instance vektorový čas metody $m_{event}$, který bude značit začátek této instance.}
\item{Vytvořit mapu parametrů $parameters$ ze sekvence metod šablony.}
\item{Přiřadit hodnoty parametrů z metody $m_{event}$ k odpovídajícím klíčům mapy $parameters$.}
\end{itemize}

V kapitole \ref{vnitrni-rep} je zmíněno, že zástupné jména parametrů jsou uloženy v šablonách, ale jejich hodnoty v instancích. A právě poslední krok v předcházejícím výčtu se o toto uložení stará. Uložení je tedy realizováno tak, že v každé instanci existuje mapa $parameters$, kde klíče jsou zástupné jména parametrů, ze sekvence metod související šablony, a hodnoty jsou při inicializaci nastaveny na hodnotu \textit{Undefined}. Jakmile je detekována metoda $m_{event}$, pak jsou hodnoty jejích parametrů přiřazeny k patřičným klíčům v mapě $parameters$. V kontextu vytváření instance, budou hodnoty parametrů z metody $m_{event}$ vždy přiřaditelné. Ovšem pokud bude přijata jiná metoda než první, nemusejí být parametry přiřaditelné, a proto je nutné kontrolovat také jejich hodnoty (více v kapitole \ref{advance}). Pro lepší pochopení práce s mapou $parameteres$ je její inicializace vysvětlena na příkladu \ref{ex:map}. 

\begin{exmp}\label{ex:map}
Nechť existuje target se sekvencí metod: $m_1(X, Y) \quad m_2(Y, Z)$. Nechť je právě detekována událost \textit{exit} s metodou $m_1(5, 7)$. Vzhledem k tomu, že je metoda $m_1$ první metodou v sekvenci, bude vytvořena nová instance tohoto targetu podle výše uvedeného postupu. Při vytváření instance bude vytvořena mapa $parameters$, která bude mít následující obsah:
\begin{lstlisting}
paramaters = {
	"X" = Undefined,
	"Y" = Undefined,
	"Z" = Undefined
}
\end{lstlisting}

Následně budou do této mapy dosazeny hodnoty parametrů. V šabloně targetu jsou u metody $m_1$ uvedeny dva parametry se zástupnými jmény \textit{X} a \textit{Y}. Takže první hodnota (tj. 5) z detekované metody se přiřadí ke klíčí \textit{X} a druhá hodnota (tj. 7) ke klíčí \textit{Y}. Výsledkem tedy bude mapa $parameters$ s následujícím obsahem:

\begin{lstlisting}
paramaters = {
	"X" = 5,
	"Y" = 7,
	"Z" = Undefined
}
\end{lstlisting}

\end{exmp}

\subsection{Krok advance}\label{advance}

V předchozí kapitole bylo popsáno, jak probíhá vytvoření nové instance targetu nebo spoileru. V této kapitole bude, oproti tomu, ukázáno jako se instance v průběhu dynamické analýzy vyvíjí. Tento případ byl uveden jako 2. krok v kapitole \ref{sablony-instance}.

Pro zjištění, zda je splněna podmínka z kroku č. 2 z kapitoly \ref{sablony-instance}, je opět nutné porovnat metody $m_{event}$ a $m_{expect}$, přičemž toto porovnání navíc obsahuje porovnání hodnot parametrů. Všechny parametry metody $m_{event}$ musejí být \textit{shodné} nebo \textit{dosaditelné} do parametrů metody $m_{expect}$, potažmo dané instance. V předchozí kapitole \ref{new-instance} bylo vysvětleno jak jsou hodnoty parametrů v instanci uloženy a bylo zde popsáno, že všechny parametry mají po vytvoření hodnotu \textit{Undefined}. Pokud tedy metoda $m_{event}$ byla volána s určitými hodnotami parametrů, pak platí, že pro všechny její parametry $p$ mohou nastat pouze tyto případy:

\begin{enumerate}
	\item V mapě $parameters$ nemá parametr $p$ ještě přiřazenu hodnotu.
	\item Parametr $p$ má být ignorován.
	\item V mapě $parameters$ má parametr $p$ již přiřazenu hodnotu a tato hodnota je totožná s hodnotou parametru $p$.
	\item V mapě $parameters$ má parametr $p$ již přiřazenu hodnotu a tato hodnota není totožná s hodnotou parametru $p$.
\end{enumerate}

Pokud u všech parametrů metody $m_{event}$ nastanou situace z bodu 1., 2. nebo 3., pak jsou parametry považovány za \textit{shodné}. Pokud jsou navíc splněny obě podmínky z kapitoly \ref{new-instance}, pak je proveden krok \textit{advance} nad danou instancí targetu nebo spoileru.

Krok \textit{advance} umožňuje instanci $i$ posunout se vpřed k jejímu dokončení. Pokud je metoda $m_{event}$ shodná s metodou $m_{expect}$ (tj. jsou splněny výše uvedené podmínky), pak je nad danou instancí $i$ proveden jeden z následující kroků:

\begin{itemize}
\item{Pokud $m_{expect}$ je volána s parametry, z jichž všechny již mají přiřazenu totožnou hodnotu v mapě $parameters$, pak dojde k inkrementaci hodnoty atributu $actualPos$ o hodnotu 1.}
\item{Pokud $m_{expect}$ je volána s parametry, z nichž alespoň jeden nemá přiřazenu hodnotu v mapě $parameters$, pak dojde k vytvoření kopie $i_{new}$ dané instance $i$. Této nové instanci $i_new$ je přiřazen odkaz na rodičovskou instanci $i$, inkrementována hodnota atributu $actualPos$ o hodnotu 1 a aktualizována mapa $parametrs$ o nově získané hodnoty parametrů.}
\end{itemize}

Důvodem, proč se ve druhém případě vytváří nová instance, je právě dosazení parametrů. Tato situace je vysvětlena na následujícím příkladu \ref{ex2}.

\begin{exmp}\label{ex2}
Nechť je zadána definice kontraktu následovně:

\begin{lstlisting}
m1(X) m2(Y) <- { m3(X,Y) }
\end{lstlisting}

Pak se dynamický analyzátor bude snažit nalézt target se sekvencí metod \texttt{m1}, \texttt{m2}, které jsou volány s parametry \texttt{X} a \texttt{Y}. Zároveň se bude snažit nalézt spoiler, který obsahuje pouze samotnou metodu \texttt{m3}, která je ale volána s totožnými hodnotami \texttt{X} a \texttt{Y}. Na diagramu \ref{test-param-copy} je znázorněna situace, kdy je volána metoda \texttt{m2} dvakrát, pokaždé s jiným parametrem, a až druhé volání této metody vede k vytvoření targetu, který může být porušen zadaným spoilerem.

Pro popis situace je zavedeno následující označení: \texttt{m1(0)} značí vykonaní metody \texttt{m1}, s parametrem \texttt{0}, \texttt{Instance\{T; MS: m1; PAR: X=0, Y=Undefined\}} značí instanci targetu (\texttt{T}) se sekvencí metod (\texttt{MS}) a parametry (\texttt{PAR}).

V tuto chvíli bude popsáno pouze vytváření instancí targetu, nikoli spoileru, protože instance spoileru bude jediná: \texttt{Instance\{S; MS: m3; PAR: X=1, Y=2\}}. Instance targetu tedy budou vytvářeny/upravovány na základě provedených metod následovně:

\begin{enumerate}
\item \texttt{m1(1)}\\
\texttt{Instance\{T; MS: m1; PAR: X=1, Y=Undefined\}}

\item \texttt{m2(0)}\\
\texttt{Instance\{T; MS: m1; PAR: X=1, Y=Undefined\}},\\
\texttt{Instance\{T; MS: m1, m2; PAR: X=1, Y=0\}}

\item \texttt{m2(2)}\\
\texttt{Instance\{T; MS: m1; PAR: X=1, Y=Undefined\}},\\
\texttt{Instance\{T; MS: m1, m2; PAR: X=1, Y=0\}},\\
\texttt{Instance\{T; MS: m1, m2; PAR: X=1, Y=2\}}

\end{enumerate}  

V 1. kroku je vytvořena nová instance targetu. Ve 2. kroku je viditelné, že nedojde pouze ke kroku \textit{advance} nad existující instancí, ale je nejprve vytvořena kopie této instance, a až poté je vykonán krok \textit{advance} nad touto kopií. Tento krok je důležitý, protože nově vytvořená instance neodpovídá instanci spoileru, neboť má dosazeny jiné hodnoty parametrů, než se vyskytují ve spoileru. Když se v kroku 3. poté objeví vykonání metody m2 s jiným (tj. správným) parametrem, dojde opět k vytvoření kopie instance z kroku 1 a k provedení kroku \textit{advance} nad touto kopií. Tato nově vytvořená instance již odpovídá spoileru. Pokud by tedy nedošlo k vytvoření kopie instance v kroku 2, a pouze by se provedl krok \textit{advance}, pak by nebylo možné vytvořit správnou instanci ve 3 kroku.

\begin{figure}[!htbp]
    \centering
    \includegraphics[width=\textwidth]{obrazky-figures/tests/example-advance.pdf}
    \caption{Příklad demonstrující vytváření a upravování instancí.}
    \label{test-param-copy}
\end{figure}

\end{exmp}

Pokud došlo ke kroku \textit{advance}, je nutné zkontrolovat, zda instance není ukončena. Tento případ nastane, pokud hodnota atributu $actualPos$ je o 1 menší, než počet metod v sekvenci šablony. Je-li instance dokončena, je nejprve přiřazen vektorový čas události $m_{event}$, který značí čas ukončení instance. Následně jsou hledány související dokončené protějšky\footnote{Protějškem je myšlen druhý z dvojice target-spoiler. Tj. Pokud dojde k ukončení instance targetu, pak jsou hledány související ukončené spoilery; pokud dojde k ukončení instance spoileru, pak jsou hledány související ukončené targety.} a konečně, dojde-li k případu, kdy je nalezen alespoň jeden protějšek, pak je na základě vektorových hodin zkontrolováno, zda jsou tyto dvě instance konkurentní. Pokud platí i poslední podmínka, došlo k detekci porušení kontraktu.

\subsection{Vektorový čas}\label{vs-navrh}

V několika posledních kapitolách byla několikrát zmíněna práce s vektorovými hodinami, bez většího vysvětlení. Tato kapitola tedy popisuje, jakým způsobem je navržena práce s těmito hodinami. Vektorové hodiny jsou důležité pro určení zda jsou dvě instance targetu a spoileru konkurentní, tj. zda se mohou vyskytnout v takovém proložení, aby došlo k jejich porušení (více viz \ref{contract-all}).

Princip práce s vektorovým časem, který je použitý v dynamickém analyzátoru kontraktů, vychází z DJIT+ algoritmu \cite{cite:VC-anaconda}. Vektorový čas je udržován v každém vláknu a objektu, který je použitý jako zámek. V nástroji Roadrunner jsou pro jeho použití využity následující metody:

\begin{description}
\item[Create] V této metodě dochází k vytvoření instance vektorových hodin pro nové vlákno.
\item[PreStart] V této metodě dochází k inkrementaci vektorových hodin obou vláken, tj. rodičovského i nově vytvořeného vlákna.
\item[Acquire] Zde je aktualizován vektorový čas vlákna pomocí vektorového času zámku\footnote{Detailnější popis aktualizací vektorových časů je popsán ve zmíněném článku \cite{cite:VC-anakonda}.}.
\item[Release] Zde je aktualizován vektorový čas zámku pomocí vektorového času vlákna. Dále je inkrementován čas vlákna v jeho vektorových hodinách.
\item[PostJoin] V této metodě dochází k aktualizaci hodin rodičovského vlákna pomocí vektorového času ukončovaného vlákna. Dále proběhne inkrementace vektorových hodin rodičovského vlákna.
\end{description}

%Při každém vytvoření instance targetu nebo spoileru je této instanci přiřazen vektorový čas vlákna. Tato první časová značka označuje čas výskytu první metody, a tedy čas začátku kontraktu. Dále je vektorový čas rovněž ukládán při ukončení každé instance. Díky těmto dvou časovým značkám (začátku a konce) mohou být instance porovnávány a může být zjišťováno, zda nedošlo k porušení kontraktu.

%potřeba zachytávat události: (init - rozparsování kontraktu), exit, acquire, release, makeShadowVar, create(creat CV), preStart(tick), postJoin

%shadowthread je odekorovano vektorovymi hodinami stejne jako shadow lock. Vychazi to z DJIT+ algoritmu.

%shadowVar kazdeho objektu ma objectsMapping( zde je mapovani shadowThread a Window, tj. pro kazde vklakno je zvlast okno, ktere obsahuje pouze metody provedene nad timto urcitym objektem)

%Window obsahuje vsechny instance targetu a spoileru pro dany objekt

%Struktura contractu, targetu, spoileru, co je ten "template", kde se ukladaji parametry

\subsection{Informace o detekci porušení}

V předchozích kapitolách bylo popsáno, jakým způsobem dojde k detekci porušení. V této kapitole jsou pouze definovány důležité parametry, které je potřeba sdělit na výstupu. Jako důležité informace jsou tedy vybrány:
\begin{itemize}
\item{Metoda, po které došlo k detekci porušení.}
\item{Vlákno, ve kterém byla metoda vyvolána.}
\item{Stacktrace vlákna, ve kterém byla metoda vyvolána.}
\item{Instance targetu, který byl porušen.}
\item{Instance spoileru, které jej porušil.}
\end{itemize}

\section{Podpora nástroje v SearchBestie}
V kapitole \ref{navrh-rr-nastroj} byl popsán návrh nástroje \textit{ContractTool} ve frameworku Roadrunner. Tato kapitola popisuje návrh částí systému, díky kterým je možné tento nástroj používat z platformy SearchBestie. První částí je vytvoření tzv. \textit{cover produceru} pro analyzátor kontraktů. Další částí je úprava nástroje RRPlugin a poslední částí je vytvoření tzv. \textit{result item creatoru}\footnote{Komunikace mezi těmito částmi byla popsána v kapitole \ref{prehled-casti-systemu}}. Kromě vytvoření těchto částí je třeba navrhnout reprezentaci parametrů pro analyzátor kontraktů, které budou předány v konfiguračním souboru pro platformu SearchBestie.

\subsection{Cover producer a úprava nástroje RRPlugin}\label{cover-producer-rr}

\textit{ContractCoverProducer} (CCP) má za úkol sbírat informace o porušení kontraktů a po skončení testování tyto informace uložit do výstupního souboru. Tato třída je stejně jako \textit{ContractTool} součástí projektu Roadrunner a chyby jsou reportovány přímo z tohoto nástroje. K reportování chyb ovšem dochází pouze v případě, že je analyzátor spouštěn z prostředí SearchBeastie, tj. kromě nástroje ContractTool je v řetězci nástrojů také RRPlugin, který vytváří instanci této třídy. Pokud RRPlugin není v řetězci nástrojů při spuštění RoadRunneru, nedojde k vytvoření instance CCP a \textit{ContractTool} vypisuje nalezené chyby pouze na výstup.

Struktura CCP je znázorněna na diagramu \ref{class-cover}. Důležité jsou především metody \textit{reportContractViolation} a \textit{writeResultToFile}. První zmíněná metoda slouží pro zaznamenání porušení kontraktu a druhá metoda slouží pro vytvoření výstupního souboru a zapsání výsledku. CCP si zaznamenává počet porušení jednotlivých targetů, které jsou identifikované pomocí přiděleného identifikátoru. Obsah výstupního souboru je navržen jako posloupnost čísel oddělených dvojtečkou:
$$x_0 : x_1 : ... : x_{n-1}$$
kde $x_i$ značí počet nalezených porušení targetu s identifikátorem $i$ a $n$ značí počet všech targetů v kontraktu. Takto formátovaný výstupní soubor je později zpracován třídou ContractRICreator, která je popsána v kapitole \ref{ricreator}.

Výše popsaný CCP je vytvářen v nástroji RRPlugin tak, jako ostatní cover producery. Z tohoto důvodu musí být CCP také korektně ukončen v momentě, kdy dochází k ukončení nástroje RRPlugin. Proces vytvoření, běhu a ukončení CCP je znázorněn na diagramu \ref{komunikace-contract}.
  
\begin{figure}[!htbp]
    \includegraphics[scale=1]{obrazky-figures/class-diagram-cover-producer.pdf}
    \caption{Diagram zobrazující třídu ContractCoverProducer.}
    \label{class-cover}
\end{figure}  

\subsection{ContractRICreator a úprava SearchBestie}\label{ricreator}
Pro podporu analyzátoru kontraktů v platformě SearchBestie je nejprve třeba definovat nové parametry do vstupního XML konfiguračního souboru. Těmito novými parametry jsou:

\begin{itemize}
\item \texttt{<parameter key="contractFile"/>} -- Parametr definující cestu ke konfiguračnímu souboru kontraktu.
\item \texttt{<parameter key="CTcontract"/>} -- Parametr povolující vytvoření cover produceru ContractCoverProducer.
\end{itemize} 

Tyto parametry jsou vnořeny do elementu \texttt{parameters}, díky čemuž jsou automaticky parsovány. Parametry, jejichž klíč začínající na prefix \textit{CT}, jsou považovány za parametry povolující cover producery a ostatní parametry jsou předávány Roadrunneru. Proto musí být klíč shodný jako parametr definovaný v kapitole \ref{config-file}. Pro parametr \textit{CTcontract} je ještě nutné přidat hodnotu výčtového typu (CONTRACT(...)) do třídy \textit{ConcurrencyCoverage}\footnote{Tato hodnota definuje prefixy pro výstupní soubory, názvy složek atd.}. Díky tomuto kroku je možné upravit fitness funkci, která se používá pro ohodnocení jednoho běhu například takto:
 
\begin{lstlisting}
<fitness class="cz.vutbr.fit.sbestie.search.fitness.FitnessExpression" name="MyFitness">
	<parameters>
		<parameter key="expression" value="CONTRACT" />
	</parameters>
</fitness>
\end{lstlisting}

Takto definovaná fitness funkce se bude snažit maximalizovat počet nalezených porušení kontraktů porušení.

Posledním krokem je vytvoření ContractResultItemCreatoru. Tato třída slouží ke zpracování výsledků z RoadRunneru (konkrétně z ContractCoverProduceru) a vytvoření výsledku, se kterým dokáže pracovat SearchBestie. Jako tento výsledek byla použita třída IntResultItem, která reprezentuje výsledek jednoho testu pouze číselnou hodnotou. ContractResultItemCreator tedy zpracuje sekvenci hodnot reprezentující počty nalezených porušení jednotlivých targetů, tyto výsledky sečte a vytvoří instanci třídy IntResultItem, do které uloží tuto sumu. Tento výsledek bude tedy reprezentovat počet všech nalezených porušení kontraktu, čehož je využito ve fitness funkci definované v předchozí kapitole.

Na diagramu je znázorněna struktura třídy ContractResultItemCreator, která rozšiřuje třídu třídu \textit{ResultItItemCreator}, a je tedy nutné, zařadit výše popsanou funkcionalitu do správných metod této nadtřídy. Důležité jsou zejména metody \textit{acceptEvent} a \textit{afterTest}. První jmenovaná metoda slouží pro uložení názvu podsložky\footnote{V případě hledání porušení kontraktů je podsložka vždy \texttt{contract/}. Název této složky je definován ve výčtu \textit{ConcurrencyCoverage} zmíněněm výše.} a identifikátoru testu, který byl spuštěn. Druhá metoda je zavolána po skončení testu a je využita pro zpracování výstupního souboru z RoadRunneru. Výstupem této metody je popsaný číselný výsledek testu, reprezentovaný instancí třídy \textit{IntResultItem}.

\begin{figure}[!htbp]
    \includegraphics[width=\textwidth]{obrazky-figures/class-diagram-RI.pdf}
    \caption{Diagram zobrazující třídu ContractResultItemCreator.}
    \label{class-cover}
\end{figure}

\chapter{Implementace}

Tato kapitole popisuje implementační detaily a řešení problémů, které při implementaci vznikly. Implementace přímo vychází z předchozí kapitoly \ref{navrh}.

Porovnávání parametrů ??

\section{Implementační problémy}

Během implementace se vyskytlo několik problémů, které musely být nutně vyřešeny, jinak by nebylo možné tuto práci dokončit. Tato kapitola tedy popisuje tyto problémy a jejich řešení.

Při návrhu dynamického analyzátoru ve frameworku RoadRunner bylo předpokládáno, že v metodách \textit{enter} a \textit{exit} jsou k dispozici hodnoty parametrů, se kterými byly metody zavolány, potažmo návratová hodnota metody. V průběhu implementace bylo ale zjištěno, že objekty, které jsou předány do zmíněných metod, obsahují pouze statické informace o těchto parametrech, tj. počet parametrů a jejich datové typy. Bylo tedy nutné upravit RoadRunner tak, aby hodnoty parametrů byly v metodách enter a exit k dispozici. Úpravy, které byly provedeny jsou posány v následujících kapitolách \ref{impl-params} a \ref{impl-ret-param}. 

\subsection{Rozšíření RoadRunneru o parametry metod}\label{impl-prams}

RoadRunner reprezentuje událost vstupu (i výstupu) objektem třídy MethodEvent. Instance této třídy je předána jako parametr, jak do metody \textit{enter}, tak i do metody \textit{exit} třídy Tool. Je tedy nutné rozšířit třídu MethodEvent o atribut, který obsahuje hodnoty parametrů. Tento parametr byl nazván \texttt{params} a je typu \texttt{Object[]}. Důvod, proč byl zvolen tento typ je, že v Javě se mohou vyskytovat jak primitivní, tak referenční datové typy. Všechny referenční datové typy rozšiřují třídu Object, takže mohou být v tomto poli uloženy. Primitivní datové typy třídu object nerozšiřují, nicméně, ke každému primitivnímu datovému typu existuje referenční datový typ. Převod primitivního datového typu na referenční se provádí pomocí metody \textit{valueOf} daného referenčního typu. Pokud tedy mají být uloženy všechny datové typy do jednoho pole, je nutné všechny typy převést na referenční a tyto referenční datové typy uložit do tohoto pole. Výsledkem tedy je pole, obsahující jak původní referenční datové typy, tak původně primitivní datové typy.

Objekty třídy MethodEvent jsou generovány ve třídě \textit{RREventGenerator}, konkrétně v metodě \textit{enter}. Do této metody je nutné přidat další parametr \texttt{Object[] params}, ve kterém budou zaslány hodnoty parametrů. Uvnitř této metody probíhá vytvoření instance třídy \textit{MethodEnter}, které je třeba přiřadit hodnoty parametrů z parametru \texttt{params}. Tato metoda je vyvolávána staticky dle její deklarace ve třídě \textit{Constant}. Tento záznam je tedy třeba modifikovat a přidat parametr \texttt{params}. Vyvořená instance třídy \textit{MethodEvent} je předána prvnímu nástroji v řetězci nástrojů v metodě \textit{enter}. Zjednodušeně bude tedy upravená metoda \textit{enter} třídy \textit{RREventGenerator} vypadat následovně:

\begin{lstlisting}
public static void enter(final Object target, final int methodDataId, final ShadowThread td, Object[] params) {
	...
	final MethodEvent me = td.enter(target, methodData);
	me.setParams(params);
	...
	firstTool.enter(me);

}
\end{lstlisting}

Dalším krokem je získání hodnot parametrů, se kterými byla zavolána metoda v testovaném programu. K tomuto kroku je nutné nejprve uvést, jakým způsobem Roadrunner získává informace o testovaném programu. RoadRunner využívá knihovnu ASM\footnote{Knihovna ASM je dostupná na adrese \url{http://asm.ow2.org/}.}, která slouží pro manipulaci a analýzu Java bytekódu. Konkrétně se analýza a manipulace s metodami provádí ve třídě SyncAndMethodThunkInserter v metodě \textit{createMethodThunk}, kde je dostupný kontext vláken testovaného program (tj. zásobník, registry, atd.) prostřednictvím objektu \texttt{mv} typu \textit{MethodVisitor}\footnote{MethodVisitor je třída z knihovny ASM, sloužící k analýze a manipulaci s bytekódem souvisejícím s metodami.}. Odtud je dále volána metoda \textit{enter} třídy \textit{RREventGenerator}, přičemž k jejímu volání dochází ihned po vstupu do metody v testovaném programu. Metoda \textit{createMethodThunk} třídy \textit{SyncAndMethodThunkInserter} byla původně volána následujícím způsobem:

\begin{lstlisting}[frame=none, numbers=left, xleftmargin=1cm]
private void createMethodThunk(int access, String name, String desc, String signature, String[] exceptions, String wrappedMethodName, int maxLocals) {
	...
	MethodInfo m = method;
	mv.push(m.getId());
	mv.invokeStatic(Constants.THREAD_STATE_TYPE, Constants.CURRENT_THREAD_METHOD);
	mv.invokeStatic(Constants.MANAGER_TYPE, Constants.ENTER_METHOD);
	...
}
\end{lstlisting}

Ve výše uvedeném kódu proběhne na řádku 4 uložení methodDataId na zásobník. Na řádku 5 dojde k vyvolání statické metody \textit{getCurrentShadowThread}, definované ve třídě Constants, která uloží na vrchol zásobníku objekt ShadowThread asiciovaný k aktuálnímu vláknu. Jako poslední proběhne na řádku 6 vyvolání statické metody \textit{enter} třídy \textit{RREventGenerator}. Pro správné vyvolání této metody, je nutné mít umístěny všechny její parametry na zásobníku, což ale není splněno, jelikož byl do této metody přidán parametr \texttt{Object[] params}. Z tohoto důvodu je mezi 5. a 6. řádek programu vložen kód, který na zásobník umístí pole obsahující hodnoty parametrů metody z testovaného programu. Upravená metoda \textit{createMethodThunk} je zobrazena v \ref{opcode-1}:

\begin{lstlisting}[frame=none, numbers=left, xleftmargin=1cm, label=opcode-1, caption=Upravená metoda createMethodThunk zajišťující vytvoření pole hodnot parametrů volané metody.]
private void createMethodThunk(int access, String name, String desc, String signature, String[] exceptions, String wrappedMethodName, int maxLocals) {
	...
	MethodInfo m = method;
	mv.push(m.getId());
	mv.invokeStatic(Constants.THREAD_STATE_TYPE, Constants.CURRENT_THREAD_METHOD);
	
	//////////////////////////////////////////////////////////////////////////////
	// ZACATEK: vlozeny kod vytvarejici pole hodnoty parametru
	//////////////////////////////////////////////////////////////////////////////
	
	Type[] paramTypes = Type.getArgumentTypes(method.getDescriptor());
	int paramLength = paramTypes.length;

	if(paramLength > 0) {
		
		// ulozeni velikosti pole parametru na zasobnik	
		mv.visitIntInsn(Opcodes.BIPUSH, paramLength);
		
		// vytvoreni pole params o velikosti umistene na vrcholu zasobniku
		mv.visitTypeInsn(Opcodes.ANEWARRAY, "java/lang/Object");
		
		// ulozeni tohoto pole do locals
		mv.visitVarInsn(Opcodes.ASTORE, paramLength + PARAM_OFFSET);

		Integer i = new Integer(0);
		
		// cyklus pres vsechny parametry dle jejich typu
		for (Type type : paramTypes) {
			
			// ulozeni pole params na vrchol zasobniku
			mv.visitVarInsn(Opcodes.ALOAD, paramLength + PARAM_OFFSET);
			
			// ulozeni idenxu na vrchol zasobniku
			mv.visitIntInsn(Opcodes.BIPUSH, i);
			
			// zapouzdreni primitivnich typu a zduplikovani hodnoty i-teho parametru
			// na vrchol zasobniku
			i = boxAndVisitVariable(mv, i, type);

			// ulozeni hodnoty parametru do pole params na index i 
			mv.visitInsn(Opcodes.AASTORE);
			i++;
		}
		
		// ulozeni pole params na vrchol zasobniku			
		mv.visitVarInsn(Opcodes.ALOAD, paramLength + PARAM_OFFSET);
	} else {
		// ulozeni hodnoy NULL na vrchol zasobniku
		mv.visitInsn(Opcodes.ACONST_NULL);
	}
	
	//////////////////////////////////////////////////////////////////////////////
	// KONEC: vlozeny kod vytvarejici pole hodnoty parametru
	//////////////////////////////////////////////////////////////////////////////
	
	mv.invokeStatic(Constants.MANAGER_TYPE, Constants.ENTER_METHOD);
	...
}
\end{lstlisting}

V kódu výše je metoda \texttt{boxAndVisitVariable}, která provádí zapouzdření primitivních typů do referenčních datových typů. Hodnoty parametrů metody v testovaném programu jsou umístěny v \textit{locals} a proto tento převod musí probíhat opět na úrovni Java bytekódu. Kód této metody a je znázorněn v příkladu \ref{opcode-2}. Pokud hodnota parametru na indexu $i$ je primitivního typu, pak tento kód načte tuto hodnotu z locals a uloží ji na vrchol zásobníku. Poté zavolá metodu \textit{valueOf} odpovídajícího datového typu, která vytvoří hodnotu referenčního datového typu z hodnoty na vrcholu zásobníku a výsledek umístí opět na vrchol zásobníku. Pokud již parametr je referenčního datového typu, pak je pouze načten z locals na vrchol zásobníku. 

\begin{lstlisting}[frame=none, numbers=left, xleftmargin=1cm, label=opcode-1, caption=Metoda zajišťující zapouzdření primitivních datových typů.]
private Integer boxAndVisitVariable(RRMethodAdapter mv, Integer i, Type type) {
		if (type.equals(Type.BOOLEAN_TYPE)) {
			mv.visitVarInsn(Opcodes.ILOAD, i + 1);
			mv.visitMethodInsn(Opcodes.INVOKESTATIC, "java/lang/Boolean", "valueOf", "(Z)Ljava/lang/Boolean;", false);
		} else if (type.equals(Type.BYTE_TYPE)) {
			mv.visitVarInsn(Opcodes.ILOAD, i + 1);
			mv.visitMethodInsn(Opcodes.INVOKESTATIC, "java/lang/Byte", "valueOf", "(B)Ljava/lang/Byte;", false);
		} else if (type.equals(Type.CHAR_TYPE)) {
			mv.visitVarInsn(Opcodes.ILOAD, i + 1);
			mv.visitMethodInsn(Opcodes.INVOKESTATIC, "java/lang/Character", "valueOf", "(C)Ljava/lang/Character;", false);
		} else if (type.equals(Type.SHORT_TYPE)) {
			mv.visitVarInsn(Opcodes.ILOAD, i + 1);
			mv.visitMethodInsn(Opcodes.INVOKESTATIC, "java/lang/Short", "valueOf", "(S)Ljava/lang/Short;", false);
		} else if (type.equals(Type.INT_TYPE)) {
			mv.visitVarInsn(Opcodes.ILOAD, i + 1);
			mv.visitMethodInsn(Opcodes.INVOKESTATIC, "java/lang/Integer", "valueOf", "(I)Ljava/lang/Integer;", false);
		} else if (type.equals(Type.LONG_TYPE)) {
			mv.visitVarInsn(Opcodes.LLOAD, i + 1);
			mv.visitMethodInsn(Opcodes.INVOKESTATIC, "java/lang/Long", "valueOf", "(J)Ljava/lang/Long;", false);
			i++;
		} else if (type.equals(Type.FLOAT_TYPE)) {
			mv.visitVarInsn(Opcodes.FLOAD, i + 1);
			mv.visitMethodInsn(Opcodes.INVOKESTATIC, "java/lang/Float", "valueOf", "(F)Ljava/lang/Float;", false);
		} else if (type.equals(Type.DOUBLE_TYPE)) {
			mv.visitVarInsn(Opcodes.DLOAD, i + 1);
			mv.visitMethodInsn(Opcodes.INVOKESTATIC, "java/lang/Double", "valueOf", "(D)Ljava/lang/Double;", false);
			i++;
		} else {
			mv.visitVarInsn(Opcodes.ALOAD, i + 1);
		}
		return i;
	}
\end{lstlisting}

Výše uvedeným způsobem je docíleno zduplikování hodnot parametrů volané metody a jejich postupnému předání až do metody \textit{enter} jednotlivých nástrojů. Dálším krokem je obdobným způsobem získat a předat návratovou hodnotu z metod testovaného programu.

\subsection{Rozšíření RoadRunneru o návratový parametr}\label{impl-ret-pram}

Rozšíření o návratovou hodnotu metod je provedeno stejným způsobem jako rozšíření a parametry popsaný v předchozí kapitole \ref{impl-prams}.

Nejprve je třeba přidat atribut \texttt{Object returnValue} do třídy MethodEvent. Dalším krokem je přidat stejný parametr do metody \textit{exit} ve třídě \textit{RREventGenerator}, ve které proběhne (stejně jako v metodě \textit{enter} téže třídy) vytvoření instance třídy \textit{MethodEvent} a přiřazení parametru \texttt{returnValue} této instanci. Tato metoda je stejně jako metoda \textit{enter} volána z metody \textit{createMethodThunk}, kde musí být návratová hodnota také stejným způsobem získána. Vzhledem k tomu, že kód získání návratové hodnoty je téměř stejný jako kód získání ostatních parametrů, není zde uveden, nicméně je dohledatelný ve zdrojových souborech.


Psat to "je to vyreseno"
[Poznamky z konzultace] Implementacni detaily - jak je to vyreseno a problemy ktere vznikly. Strukturalni a behavioralni diagramy
Co bylo zajimave, co sem jak musel vyresit kdyz nastaly nejake problemy

-implementacni problemy
- v cem je to implementovane, ceho se vyuziva
- jak se spousti (priklady)


\section{Implementační detaily}


\chapter{Testování}

\section{Jednotkové testování}

\section{Testování dynamického analyzátoru ve frameworku RoadRunner}

Následující testy jsou popsány jednotným způsobem, a proto bude tato reprezentace nejprve vysvětlena. Každý test se skládá z textového popisu a sekvenčního diagramu. Součástí textového popisu jsou následující položky:
\textit{kategorie}, \textit{popis}, \textit{kontrakt}, \textit{očekávaný} a \textit{skutečný výsledek}. \textit{Kategorie} udává na co se daný test zaměřuje, přičemž u jednoho testu může být kategorií více. Kategorie jsou:

\begin{description}
\item{\textit{JEDNODUCHÝ KONTRAKT} -- Testování probíhá pouze na jednoduchých kontraktech.}
\item{\textit{SLOŽITĚJŠÍ KONTRAKT} -- Testování probíhá pouze na složitějších kontraktech.}
\item{\textit{VEKTOROVÝ ČAS} -- K odhalení porušení je nutné použití vektorového času.}
\item{\textit{VÍCE OBJEKTŮ} -- Metody targetů a spoilerů jsou volány na více objektů stejných typů.}
\item{\textit{VÍCE SPOILERŮ} -- K jednomu targetu je definováno více spoilerů.}
\item{\textit{VÍCE VLÁKEN} -- Metody targetl a spoilerů jsou volány z více vláken.}
\item{\textit{IGNOROVÁNÍ OSTATNÍCH METOD} -- V programu se nachází volání metod, které je nutné ignorovat.}
\item{\textit{NEDOKONČENÉ SEKVENCE} -- V programu se vyskytují nedokončené sekvence metod.}
\item{\textit{PARAMETRY} -- Kontrakty jsou zadány s parametry.}
\item{\textit{RŮZNÉ TYPY PARAMETRŮ} -- V metodách se vyskytují parametry různých typů.}
\item{\textit{IGNOROVÁNÍ PARAMETRŮ} -- V kontraktu je zadán parametr \texttt{\_}, která je nutné ignorovat.}
\item{\textit{VÍCE PARAMETRŮ} -- V metodách se vyskytuje více parametrů.}
\item{\textit{NÁVRATOVÝ PARAMETR} -- V kontraktu je zadán návratový parametr.}
\end{description}

\textit{Popis} uvádí detailnější informace o důvodu porušení (nebo neporušení) kontraktu. Položka \textit{kontrakt} uvádí definici kontraktu, který je hledán a poslední dvě položky (\textit{očekávaný} a \textit{skutečný výsledek}) uvádějí, kolik porušení kontraktů mělo být nalezeno a kolik jich bylo skutečné nalezeno.

Na sekvenčním diagramu je znázorněna testovaná situace. Vyskytují se zda vlákna targetů (\textit{Target thread n}), vlákna spoilerů (Spoiler thread n) a objekty (Object n), na které jsou jednotlivé metody volány\footnote{Písmeno n v názvu vláken a objektů značí jejich číslo}. Metody objektů jsou volány z vláken a to buď synchronizovaně nebo bez synchronizace. Synchronizovaným voláním\footnote{Synchonizované a nesynchronizované volání metody je na diagramech rozlišeno pomocí typu šipky a také klíčovým slovem synchronized.} je myšleno volání metod zapouzdřené do vhodné synchronizace, tj. uzamknutí zámku před voláním první metody v sekvenci a uvolnění zámku po provedení poslední metody v sekvenci. Takovéto volání metod je znozorněno na digramu ...TODO.... Pokud je navrácena návratová hodnota v takto prováděné sekvenci metod, pak je to znázorněno pomocí popisu hrany. Tj. například v příklad ...TODO... je návratová hodnota metody \texttt{m2} hodnota 5, což je zapsáno jako: \texttt{5 = m2()} ...TODO upravit...
Volání metody bez použití synchronizace je zobrazena na diagramu ...TODO... Pokud dojde k takovému volání, není zaručeno pořadí volání mezi voláním metod z různých vlákne. Jediná situace, kdy je zaručeno pořadí vykonávání metod napříč různými vlákny je tedy případ, kdy obě vlákna použijí synchronizaci nad stejným objektem (tj. zámkem).

V příkladech se také vyskytuje uspaní vláken na určitou dobu z důvodu nasimulování různého proložení (volání \textit{sleep}).



\subsection*{Test 1}

\begin{itemize}
\item Kategorie: JEDNODUCHÝ KONTRAKT.
\item Popis: Obě vlákna využívají korektní synchronizaci.
\item Kontrakt: \texttt{m1() m2() <- { m1() }}
\item Očekávaný výsledek: 0 nalezených porušení.
\item Skutečný výsledek: 0 nalezených porušení.
\item \textbf{Test proběhl úspěšně}.
\end{itemize}

\begin{center}
    \centering
    \includegraphics[width=\textwidth]{obrazky-figures/tests/test1.pdf}
    %\caption{Test 1.}
    \label{test1}
\end{center}

\subsection*{Test 2}

\begin{itemize}
\item Kategorie: JEDNODUCHÝ KONTRAKT.
\item Popis: Vlákno $S$ volá metodu \textit{m1} bez synchronizace.
\item Kontrakt: \texttt{m1() m2() <- { m1() }}
\item Očekávaný výsledek: 1 nalezených porušení.
\item Skutečný výsledek: 1 nalezených porušení.
\item \textbf{Test proběhl úspěšně}.
\end{itemize}

\begin{center}
    \centering
    \includegraphics[width=\textwidth]{obrazky-figures/tests/test2.pdf}
    %\caption{Test 2.}
    \label{test2}
\end{center}

\subsection*{Test 3}
\begin{itemize}
\item Kategorie: JEDNODUCHÝ KONTRAKT.
\item Popis: Vlákno $T$ volá metody \textit{m1} a \textit{m2} bez synchronizace.
\item Kontrakt: \texttt{m1() m2() <- { m1() }}
\item Očekávaný výsledek: 1 nalezených porušení.
\item Skutečný výsledek: 1 nalezených porušení.
\item \textbf{Test proběhl úspěšně}.
\end{itemize}

\begin{center}
    \centering
    \includegraphics[width=\textwidth]{obrazky-figures/tests/test3.pdf}
    %\caption{Test 3.}
    \label{test3}
\end{center}

\subsection*{Test 4}
\begin{itemize}

\item Kategorie: JEDNODUCHÝ KONTRAKT.
\item Popis: Obě vlákna volají všechny metody bez synchronizace.
\item Kontrakt: \texttt{m1() m2() <- { m1() }}
\item Očekávaný výsledek: 1 nalezených porušení.
\item Skutečný výsledek: 1 nalezených porušení.
\item \textbf{Test proběhl úspěšně}.
\end{itemize}

\begin{center}
    \centering
    \includegraphics[width=\textwidth]{obrazky-figures/tests/test4.pdf}
    %\caption{Test 4.}
    \label{test4}
\end{center}

\subsection*{Test 5}
\begin{itemize}

\item Kategorie: JEDNODUCHÝ KONTRAKT, VEKTOROVÝ ČAS.
\item Popis: Obě vlákna volají všechny metody bez synchronizace. Pomocí uspání vláken, je spoiler v čase vykonán mezi první a poslední metodou targetu. K detekování není nutný vektorový čas.
\item Kontrakt: \texttt{m1() m2() <- { m1() }}
\item Očekávaný výsledek: 1 nalezených porušení.
\item Skutečný výsledek: 1 nalezených porušení.
\item \textbf{Test proběhl úspěšně}.
\end{itemize}

\begin{center}
    \centering
    \includegraphics[width=\textwidth]{obrazky-figures/tests/test5.pdf}
    %\caption{Test 5.}
    \label{test5}
\end{center}

\subsection*{Test 6}
\begin{itemize}
\item Kategorie: JEDNODUCHÝ KONTRAKT, VEKTOROVÝ ČAS.
\item Popis: Obě vlákna volají všechny metody bez synchronizace. Pomocí uspání vláken, je spoiler v čase vykonán dříve, než je vykonána první metoda targetu. K detekování je nutný vektorový čas.
\item Kontrakt: \texttt{m1() m2() <- { m1() }}
\item Očekávaný výsledek: 1 nalezených porušení.
\item Skutečný výsledek: 1 nalezených porušení.
\item \textbf{Test proběhl úspěšně}.
\end{itemize}

\begin{center}
    \centering
    \includegraphics[width=\textwidth]{obrazky-figures/tests/test6.pdf}
    %\caption{Test 6.}
    \label{test6}
\end{center}
\subsection*{Test 7}

\begin{itemize}
\item Kategorie: JEDNODUCHÝ KONTRAKT, VEKTOROVÝ ČAS.
\item Popis: Obě vlákna volají všechny metody bez synchronizace. Pomocí uspání vláken, je celý target v čase vykonán dříve, než první metoda spoileru. K detekování je nutný vektorový čas.
\item Kontrakt: \texttt{m1() m2() <- { m1() }}
\item Očekávaný výsledek: 1 nalezených porušení.
\item Skutečný výsledek: 1 nalezených porušení.
\item \textbf{Test proběhl úspěšně}.
\end{itemize}

\begin{center}
    \centering
    \includegraphics[width=\textwidth]{obrazky-figures/tests/test7.pdf}
    %\caption{Test 7.}
    \label{test7}
\end{center}
\subsection*{Test 8}

\begin{itemize}
\item Kategorie: SLOŽITĚJŠÍ KONTRAKT, VEKTOROVÝ ČAS.
\item Popis: Obě vlákna volají všechny metody bez synchronizace.
\item Kontrakt: \texttt{m1() m2() m3() <- { m2() m3() }}
\item Očekávaný výsledek: 1 nalezených porušení.

\item Skutečný výsledek: 1 nalezených porušení.
\item \textbf{Test proběhl úspěšně}.

\end{itemize}
\begin{center}
    \centering
    \includegraphics[width=\textwidth]{obrazky-figures/tests/test8.pdf}
    %\caption{Test 8.}
    \label{test8}
\end{center}

\subsection*{Test 9}
\begin{itemize}
\item Kategorie: SLOŽITĚJŠÍ KONTRAKT, VEKTOROVÝ ČAS.
\item Popis: Vlákno $s$ volá všechny metody bez synchronizace.
\item Kontrakt: \texttt{m1() m2() m3() <- { m2() m3() }}
\item Očekávaný výsledek: 1 nalezených porušení.
\item Skutečný výsledek: 1 nalezených porušení.

\item \textbf{Test proběhl úspěšně}.
\end{itemize}
\begin{center}
    \centering
    \includegraphics[width=\textwidth]{obrazky-figures/tests/test9.pdf}
    %\caption{Test 9.}
    \label{test9}
\end{center}

\subsection*{Test 10}
\begin{itemize}
\item Kategorie: SLOŽITĚJŠÍ KONTRAKT, VEKTOROVÝ ČAS.
\item Popis: Vlákno $t$ volá všechny metody bez synchronizace.
\item Kontrakt: \texttt{m1() m2() m3() <- { m2() m3() }}
\item Očekávaný výsledek: 1 nalezených porušení.
\item Skutečný výsledek: 1 nalezených porušení.

\item \textbf{Test proběhl úspěšně}.
\end{itemize}
\begin{center}
    \centering
    \includegraphics[width=\textwidth]{obrazky-figures/tests/test10.pdf}
    %\caption{Test 10.}
    \label{test10}
\end{center}

\subsection*{Test 11}
\begin{itemize}
\item Kategorie: SLOŽITĚJŠÍ KONTRAKT, VEKTOROVÝ ČAS.
\item Popis: Obě vlákna volají všechny metody bez synchronizace. Pomocí uspání vláken, je target v čase vykonán mezi první a poslední metodou spoileru. K detekování je nutný vektorový čas.
\item Kontrakt: \texttt{m1() m2() m3() <- { m2() m3() }}

\item Očekávaný výsledek: 1 nalezených porušení.
\item Skutečný výsledek: 1 nalezených porušení.
\item \textbf{Test proběhl úspěšně}.
\end{itemize}
\begin{center}
    \centering
    \includegraphics[width=\textwidth]{obrazky-figures/tests/test11.pdf}
    %\caption{Test 11.}
    \label{test11}
\end{center}

\subsection*{Test 12}
\begin{itemize}
\item Kategorie: SLOŽITĚJŠÍ KONTRAKT, VEKTOROVÝ ČAS.
\item Popis: Vlákno $t$ volá všechny metody za použití správné synchronizace. Metoda \textit{m2} vlákna $s$ volá první metodu za použití synchronizace, ale druhou nikoli. K porušení kontraktu nedochází, neboť spoiler nemůže být celý vložen v targetu.
\item Kontrakt: \texttt{m1() m2() m3() <- { m2() m3() }}

\item Očekávaný výsledek: 0 nalezených porušení.
\item Skutečný výsledek: 0 nalezených porušení.
\item \textbf{Test proběhl úspěšně}.
\end{itemize}
\begin{center}
    \centering
    \includegraphics[width=\textwidth]{obrazky-figures/tests/test12.pdf}
    %\caption{Test 12.}
    \label{test12}
\end{center}

\subsection*{Test 13}
\begin{itemize}
\item Kategorie: SLOŽITĚJŠÍ KONTRAKT, VEKTOROVÝ ČAS.
\item Popis: Obě vlákna volají všechny metody bez synchronizace.
\item Kontrakt: \texttt{m1() m2() m3() m4() m5() <- { m1() m3() m1()}}
\item Očekávaný výsledek: 1 nalezených porušení.
\item Skutečný výsledek: 1 nalezených porušení.

\item \textbf{Test proběhl úspěšně}.
\end{itemize}
\begin{center}
    \centering
    \includegraphics[width=\textwidth]{obrazky-figures/tests/test13.pdf}
    %\caption{Test 13.}
    \label{test13}
\end{center}

\subsection*{Test 14}
\begin{itemize}
\item Kategorie: SLOŽITĚJŠÍ KONTRAKT, VEKTOROVÝ ČAS.
\item Popis: Vlákno $S$ volá všechny metody bez synchronizace.
\item Kontrakt: \texttt{m1() m2() m3() m4() m5() <- { m1() m3() m1()}}
\item Očekávaný výsledek: 1 nalezených porušení.
\item Skutečný výsledek: 1 nalezených porušení.

\item \textbf{Test proběhl úspěšně}.
\end{itemize}
\begin{center}
    \centering
    \includegraphics[width=\textwidth]{obrazky-figures/tests/test14.pdf}
    %\caption{Test 14.}
    \label{test14}
\end{center}

\subsection*{Test 15}
\begin{itemize}
\item Kategorie: SLOŽITĚJŠÍ KONTRAKT, VEKTOROVÝ ČAS.
\item Popis: Vlákno $T$ volá všechny metody bez synchronizace.
\item Kontrakt: \texttt{m1() m2() m3() m4() m5() <- { m1() m3() m1()}}
\item Očekávaný výsledek: 1 nalezených porušení.
\item Skutečný výsledek: 1 nalezených porušení.

\item \textbf{Test proběhl úspěšně}.
\end{itemize}
\begin{center}
    \centering
    \includegraphics[width=\textwidth]{obrazky-figures/tests/test15.pdf}
    %\caption{Test 15.}
    \label{test15}
\end{center}
\subsection*{Test 16}
\begin{itemize}
\item Kategorie: IGNOROVÁNÍ OSTATNÍCH METOD.
\item Popis: Ve vlákně $T$ i $S$ vznikne více instancí targetů a spoilerů. Testuje se ignorování ostatních metod.
\item Kontrakt: \texttt{m1() m2() m3() m4() m5() <- { m1() m3() m1()}}
\item Očekávaný výsledek: 4 nalezených porušení.

\item Skutečný výsledek: 4 nalezených porušení.
\item \textbf{Test proběhl úspěšně}.
\end{itemize}
\begin{center}
    \centering
    \includegraphics[width=\textwidth]{obrazky-figures/tests/test16.pdf}
    %\caption{Test 16.}
    \label{test16}
\end{center}

\subsection*{Test 17}
\begin{itemize}
\item Kategorie: NEDOKONČENÉ SEKVENCE.
\item Popis: Target ve vlákně $T$ není dokončen - chybí poslední metoda.
\item Kontrakt: \texttt{m1() m2() m3() m4() m5() <- { m1() m3() m1()}}
\item Očekávaný výsledek: 0 nalezených porušení.
\item Skutečný výsledek: 0 nalezených porušení.

\item \textbf{Test proběhl úspěšně}.
\end{itemize}
\begin{center}
    \centering
    \includegraphics[width=\textwidth]{obrazky-figures/tests/test17.pdf}
    %\caption{Test 17.}
    \label{test17}
\end{center}
\subsection*{Test 18}
\begin{itemize}
\item Kategorie: NEDOKONČENÉ SEKVENCE.
\item Popis: Spoiler ve vlákně $S$ není dokončen - chybí první metoda.
\item Kontrakt: \texttt{m1() m2() m3() m4() m5() <- { m1() m3() m1()}}
\item Očekávaný výsledek: 0 nalezených porušení.
\item Skutečný výsledek: 0 nalezených porušení.

\item \textbf{Test proběhl úspěšně}.
\end{itemize}
\begin{center}
    \centering
    \includegraphics[width=\textwidth]{obrazky-figures/tests/test18.pdf}
    %\caption{Test 18.}
    \label{test18}
\end{center}

\subsection*{Test 19}
\begin{itemize}
\item Kategorie: NEDOKONČENÉ SEKVENCE.
\item Popis: Spoiler ve vlákně $S$ není dokončen - chybí poslední metoda.
\item Kontrakt: \texttt{m1() m2() m3() m4() m5() <- { m1() m3() m1()}}
\item Očekávaný výsledek: 0 nalezených porušení.
\item Skutečný výsledek: 0 nalezených porušení.

\item \textbf{Test proběhl úspěšně}.
\end{itemize}
\begin{center}
    \centering
    \includegraphics[width=\textwidth]{obrazky-figures/tests/test19.pdf}
    %\caption{Test 19.}
    \label{test19}
\end{center}
\subsection*{Test 20}
\begin{itemize}
\item Kategorie:NEDOKONČENÉ SEKVENCE.
\item Popis: Spoiler ve vlákně $S$ není dokončen - chybí prostřední metoda.
\item Kontrakt: \texttt{m1() m2() m3() m4() m5() <- { m1() m3() m1()}}
\item Očekávaný výsledek: 0 nalezených porušení.
\item Skutečný výsledek: 0 nalezených porušení.

\item \textbf{Test proběhl úspěšně}.
\end{itemize}
\begin{center}
    \centering
    \includegraphics[width=\textwidth]{obrazky-figures/tests/test20.pdf}
    %\caption{Test 20.}
    \label{test20}
\end{center}

\subsection*{Test 21}
\begin{itemize}
\item Kategorie: VÍCE OBJEKTŮ.
\item Popis: Target ve vlákně $T$ je volán na objekt $o_1$, ale spoiler ve vlákně $S$ je volán na objekt $o_2$.
\item Kontrakt: \texttt{m1() m2() m3() <- { m2() m3()}}
\item Očekávaný výsledek: 0 nalezených porušení.

\item Skutečný výsledek: 0 nalezených porušení.
\item \textbf{Test proběhl úspěšně}.
\end{itemize}
\begin{center}
    \centering
    \includegraphics[width=\textwidth]{obrazky-figures/tests/test21.pdf}
    %\caption{Test 21.}
    \label{test21}
\end{center}

\subsection*{Test 22}
\begin{itemize}
\item Kategorie: VÍCE OBJEKTŮ.
\item Popis: Spoiler ve vlákně $S$ volá první metodu \textit{m2} na objekt $o_1$, ale druhou metodu \textit{m3} na objekt $o_2$.
\item Kontrakt: \texttt{m1() m2() m3() <- { m2() m3()}}
\item Očekávaný výsledek: 0 nalezených porušení.

\item Skutečný výsledek: 0 nalezených porušení.
\item \textbf{Test proběhl úspěšně}.
\end{itemize}
\begin{center}
    \centering
    \includegraphics[width=\textwidth]{obrazky-figures/tests/test22.pdf}
    %\caption{Test 22.}
    \label{test22}
\end{center}

\subsection*{Test 23}
\begin{itemize}
\item Kategorie: VÍCE OBJEKTŮ.
\item Popis: Target ve vlákně $T$ volá první metodu \textit{m2} a poslední metodu \textit{m3} na objekt $o_1$, ale druhou metodu \textit{m2} na objekt $o_2$.
\item Kontrakt: \texttt{m1() m2() m3() <- { m2() m3()}}
\item Očekávaný výsledek: 0 nalezených porušení.

\item Skutečný výsledek: 0 nalezených porušení.
\item \textbf{Test proběhl úspěšně}.
\end{itemize}
\begin{center}
    \centering
    \includegraphics[width=\textwidth]{obrazky-figures/tests/test23.pdf}
    %\caption{Test 23.}
    \label{test23}
\end{center}

\subsection*{Test 24}
\begin{itemize}
\item Kategorie: VÍCE SPOILERŮ.
\item Popis: Je provedena pouze jedna definice spoileru. Dojde k nalezení 2 porušení, neboť spoiler i target mohou být nalezeny jako ve vlákně $T$, tak i $S$. 
\item Kontrakt: \texttt{m1() m2() m3() <- { m1() m3() | m2() m4() | m5() }}
\item Očekávaný výsledek: 2 nalezených porušení.

\item Skutečný výsledek: 2 nalezených porušení.
\item \textbf{Test proběhl úspěšně}.
\end{itemize}
\begin{center}
    \centering
    \includegraphics[width=\textwidth]{obrazky-figures/tests/test24.pdf}
    %\caption{Test 24.}
    \label{test24}
\end{center}

\subsection*{Test 25}
\begin{itemize}
\item Kategorie: VÍCE SPOILERŮ.
\item Popis: Dojde k nalezení 2 různých spoilerů (\textit{m1() m3()} a \textit{m5()}) ve vlákne $S$. 
\item Kontrakt: \texttt{m1() m2() m3() <- { m1() m3() | m2() m4() | m5() }}
\item Očekávaný výsledek: 3 nalezených porušení.

\item Skutečný výsledek: 3 nalezených porušení.
\item \textbf{Test proběhl úspěšně}.
\end{itemize}
\begin{center}
    \centering
    \includegraphics[width=\textwidth]{obrazky-figures/tests/test25.pdf}
    %\caption{Test 25.}
    \label{test25}
\end{center}

\subsection*{Test 26}
\begin{itemize}
\item Kategorie: VÍCE SPOILERŮ, VÍCE VLÁKEN.
\item Popis: Dojde k nalezení 2 různých spoilerů (\textit{m1() m3()} a \textit{m5()}) ve vláknech $S1$ a $S12$. 
\item Kontrakt: \texttt{m1() m2() m3() <- { m1() m3() | m2() m4() | m5() }}
\item Očekávaný výsledek: 2 nalezených porušení.

\item Skutečný výsledek: 2 nalezených porušení.
\item \textbf{Test proběhl úspěšně}.
\end{itemize}
\begin{center}
    \centering
    \includegraphics[width=\textwidth]{obrazky-figures/tests/test26.pdf}
    %\caption{Test 26.}
    \label{test26}
\end{center}

\subsection*{Test 27}
\begin{itemize}
\item Kategorie: VÍCE SPOILERŮ, VÍCE VLÁKEN.
\item Popis: Nedojde k nalezení žádného spoileru, neboď spoiler musí být vykonán v jednom vlákně. 
\item Kontrakt: \texttt{m1() m2() m3() <- { m1() m3() | m2() m4() | m5() }}
\item Očekávaný výsledek: 0 nalezených porušení.

\item Skutečný výsledek: 0 nalezených porušení.
\item \textbf{Test proběhl úspěšně}.
\end{itemize}
\begin{center}
    \centering
    \includegraphics[width=\textwidth]{obrazky-figures/tests/test27.pdf}
    %\caption{Test 27.}
    \label{test27}
\end{center}

\subsection*{Test 28}
\begin{itemize}
\item Kategorie: VÍCE SPOILERŮ, VÍCE VLÁKEN.
\item Popis: Dojde k nalezení pouze jednoho spoileru (\textit{m2() m4()}) ve vlákně $S3$, neboť v ostatních vláknech je použita správná synchronizace. 
\item Kontrakt: \texttt{m1() m2() m3() <- { m1() m3() | m2() m4() | m5() }}
\item Očekávaný výsledek: 1 nalezených porušení.

\item Skutečný výsledek: 1 nalezených porušení.
\item \textbf{Test proběhl úspěšně}.

... TODO spatny diagram ...
\end{itemize}
\begin{center}
    \centering
    \includegraphics[width=\textwidth]{obrazky-figures/tests/test28.pdf}
    %\caption{Test 28.}
    \label{test28}
\end{center}

\subsection*{Test 29}
\begin{itemize}
\item Kategorie: PARAMETRY.
\item Popis: Parametry metod se shodují s definicí kontraktu. 
\item Kontrakt: \texttt{m1(A) m2(B) <- { m3(C) }}
\item Očekávaný výsledek: 1 nalezených porušení.
\item Skutečný výsledek: 1 nalezených porušení.

\item \textbf{Test proběhl úspěšně}.
\end{itemize}
\begin{center}
    \centering
    \includegraphics[width=\textwidth]{obrazky-figures/tests/test29.pdf}
    %\caption{Test 29.}
    \label{test29}
\end{center}

\subsection*{Test 30}
\begin{itemize}
\item Kategorie: PARAMETRY.
\item Popis: Metoda \textit{m2()} je zavolána bez parametru. 
\item Kontrakt: \texttt{m1(A) m2(B) <- { m3(C) }}
\item Očekávaný výsledek: 0 nalezených porušení.
\item Skutečný výsledek: 0 nalezených porušení.

\item \textbf{Test proběhl úspěšně}.
\end{itemize}
\begin{center}
    \centering
    \includegraphics[width=\textwidth]{obrazky-figures/tests/test30.pdf}
    %\caption{Test 30.}
    \label{test30}
\end{center}

\subsection*{Test 31}
\begin{itemize}
\item Kategorie: PARAMETRY.
\item Popis: Parametry metod \textit{m1} a \textit{m3} se shodují. 
\item Kontrakt: \texttt{m1(A) m2(B) <- { m3(A) }}
\item Očekávaný výsledek: 1 nalezených porušení.
\item Skutečný výsledek: 1 nalezených porušení.

\end{itemize}
\begin{center}
    \centering
    \includegraphics[width=\textwidth]{obrazky-figures/tests/test31.pdf}
    %\caption{Test 31.}
    \label{test31}
\end{center}

\subsection*{Test 32}
\begin{itemize}
\item Kategorie: PARAMETRY.
\item Popis: Parametry metod \textit{m1} a \textit{m3} se neshodují. 
\item Kontrakt: \texttt{m1(A) m2(B) <- { m3(A) }}
\item Očekávaný výsledek: 1 nalezených porušení.
\item Skutečný výsledek: 1 nalezených porušení.
\end{itemize}
\begin{center}
    \centering
    \includegraphics[width=\textwidth]{obrazky-figures/tests/test32.pdf}
    %\caption{Test 32.}
    \label{test32}
\end{center}
%\subsectition{Test 33on*{}

%\begin{center}
%    \centering
%    \includegraphics[width=\textwidth]{obrazky-figures/tests/test33.pdf}
%%    \caption{Test 33.}
%    \label{test33}
%\end{center}
%\subsectition{Test 34on*{}
%\begin{center}
%    \centering
%    \includegraphics[width=\textwidth]{obrazky-figures/tests/test34.pdf}
%%    \caption{Test 34.}
%    \label{test34}
%\end{center}
\subsection*{Test 35}
\begin{itemize}
\item Kategorie: RŮZNÉ TYPY PARAMETRŮ.
\item Popis: Parametry v metodách \textit{m1} a \textit{m3} jsou typu \textit{String}, oproti tomu parametr metody \textit{m2(2)} je primitivního typu \textit{int}. 
\item Kontrakt: \texttt{m1(A) m2(B) <- { m3(A) }}
\item Očekávaný výsledek: 1 nalezených porušení.

\item Skutečný výsledek: 1 nalezených porušení.
\end{itemize}
\begin{center}
    \centering
    \includegraphics[width=\textwidth]{obrazky-figures/tests/test35.pdf}
    %\caption{Test 35.}
    \label{test35}
\end{center}

\subsection*{Test 36}
\begin{itemize}
\item Kategorie: RŮZNÉ TYPY PARAMETRŮ.
\item Popis: Parametry v metodách \textit{m1} a \textit{m3} jsou typu \textit{Double}, oproti tomu parametr metody \textit{m2} je primitivního typu \textit{int}. 
\item Kontrakt: \texttt{m1(A) m2(B) <- { m3(A) }}
\item Očekávaný výsledek: 1 nalezených porušení.

\item Skutečný výsledek: 1 nalezených porušení.
\end{itemize}
\begin{center}
    \centering
    \includegraphics[width=\textwidth]{obrazky-figures/tests/test36.pdf}
    %\caption{Test 36.}
    \label{test36}
\end{center}

\subsection*{Test 37}
\begin{itemize}
\item Kategorie: RŮZNÉ TYPY PARAMETRŮ.
\item Popis: Parametry v metodách \textit{m1} a \textit{m3} jsou instance nové třídy \textit{MyObject} s definovanou \textit{equals} metodou, oproti tomu parametr metody \textit{m2} je primitivního typu \textit{int}. 
\item Kontrakt: \texttt{m1(A) m2(B) <- { m3(A) }}

\item Očekávaný výsledek: 1 nalezených porušení.
\item Skutečný výsledek: 1 nalezených porušení.
\end{itemize}
\begin{center}
    \centering
    \includegraphics[width=\textwidth]{obrazky-figures/tests/test37.pdf}
    %\caption{Test 37.}
    \label{test37}
\end{center}
\subsection*{Test 38}
\begin{itemize}
\item Kategorie: IGNOROVÁNÍ PARAMETRŮ.
\item Popis: Parametr metod \textit{m1} a \textit{m3} je ignorován, takže nezáleží na jeho hodnotě. 
\item Kontrakt: \texttt{m1(_) m2(X) <- { m3(_) }}
\item Očekávaný výsledek: 1 nalezených porušení.

\item Skutečný výsledek: 1 nalezených porušení.
\end{itemize}
\begin{center}
    \centering
    \includegraphics[width=\textwidth]{obrazky-figures/tests/test38.pdf}
    %\caption{Test 38.}
    \label{test38}
\end{center}
\subsection*{Test 39}
\begin{itemize}
\item Kategorie: IGNOROVÁNÍ PARAMETRŮ.
\item Popis: Parametr metody \textit{m1} není zadán. Hodnota a typ parametru jsou ignorovány, ale počet musí souhlasit. Tím pádem není nalezena žádná instance targetu. 
\item Kontrakt: \texttt{m1(_) m2(X) <- { m3(_) }}
\item Očekávaný výsledek: 0 nalezených porušení.

\item Skutečný výsledek: 0 nalezených porušení.
\end{itemize}
\begin{center}
    \centering
    \includegraphics[width=\textwidth]{obrazky-figures/tests/test39.pdf}
    %\caption{Test 39.}
    \label{test39}
\end{center}
\subsection*{Test 40}
\begin{itemize}
\item Kategorie: VÍCE PARAMETRŮ.
\item Popis: Parametry mezi targetem a spoilerem souhlasí, takže porušení je nalezeno. 
\item Kontrakt: \texttt{m1(A,B) m2(A,C) <- { m3(A,B,C) }}
\item Očekávaný výsledek: 1 nalezených porušení.
\item Skutečný výsledek: 1 nalezených porušení.

\end{itemize}
\begin{center}
    \centering
    \includegraphics[width=\textwidth]{obrazky-figures/tests/test40.pdf}
    %\caption{Test 40.}
    \label{test40}
\end{center}
\subsection*{Test 41}
\begin{itemize}
\item Kategorie: VÍCE PARAMETRŮ.
\item Popis: V tagetu, ve vlákně $T$, se neshoduje první parametr u metod \textit{m1} a \textit{m2}, takže neexistuje instance targetu. 
\item Kontrakt: \texttt{m1(A,B) m2(A,C) <- { m3(A,B,C) }}
\item Očekávaný výsledek: 0 nalezených porušení.

\item Skutečný výsledek: 0 nalezených porušení.
\end{itemize}
\begin{center}
    \centering
    \includegraphics[width=\textwidth]{obrazky-figures/tests/test41.pdf}
    %\caption{Test 41.}
    \label{test41}
\end{center}
\subsection*{Test 42}
\begin{itemize}
\item Kategorie: VÍCE PARAMETRŮ.
\item Popis: Dokončená instance tagetu obsahuje až druhou metodu \textit{m1}, protože u první metody \textit{m1} se neshodují parametry s metodou \textit{m2} a ani s metodou \textit{m3} ve vláknu spoileru $S$. 
\item Kontrakt: \texttt{m1(A,B) m2(A,C) <- { m3(A,B,C) }}

\item Očekávaný výsledek: 1 nalezených porušení.
\item Skutečný výsledek: 1 nalezených porušení.
\end{itemize}
\begin{center}
    \centering
    \includegraphics[width=\textwidth]{obrazky-figures/tests/test42.pdf}
    %\caption{Test 42.}
    \label{test42}
\end{center}

\subsection*{Test 43}
\begin{itemize}
\item Kategorie: VÍCE PARAMETRŮ, IGNOROVÁNÍ PARAMETRŮ, NÁVRATOVÝ PARAMETR.
\item Popis: Návratový parametr metody \textit{m1} je shodný s parametry metod \textit{m2} a \textit{m3}. 
\item Kontrakt: \texttt{A:m1(_) m2(A) <- { m3(A) }}
\item Očekávaný výsledek: 1 nalezených porušení.

\item Skutečný výsledek: 1 nalezených porušení.
\end{itemize}
\begin{center}
    \centering
    \includegraphics[width=\textwidth]{obrazky-figures/tests/test43.pdf}
    %\caption{Test 43.}
    \label{test43}
\end{center}
\subsection*{Test 44}
\begin{itemize}
\item Kategorie: VÍCE PARAMETRŮ, IGNOROVÁNÍ PARAMETRŮ, NÁVRATOVÝ PARAMETR.
\item Popis: Návratový parametr metody \textit{m1} není shodný s parametrem metody \textit{m3} ve vláknu spoileru $S$. 
\item Kontrakt: \texttt{A:m1(_) m2(A) <- { m3(A) }}
\item Očekávaný výsledek: 0 nalezených porušení.

\item Skutečný výsledek: 0 nalezených porušení.
\end{itemize}
\begin{center}
    \centering
    \includegraphics[width=\textwidth]{obrazky-figures/tests/test44.pdf}
    %\caption{Test 44.}
    \label{test44}
\end{center}
\subsection*{Test 45}
\begin{itemize}
\item Kategorie: VÍCE PARAMETRŮ, IGNOROVÁNÍ PARAMETRŮ, NÁVRATOVÝ PARAMETR.
\item Popis: Parametry jsou shodné, ale je použita správná synchronizace a proto k porušení nedojde. 
\item Kontrakt: \texttt{A:m1(_) m2(A) <- { m3(A) }}
\item Očekávaný výsledek: 0 nalezených porušení.

\item Skutečný výsledek: 0 nalezených porušení.
\end{itemize}
\begin{center}
    \centering
    \includegraphics[width=\textwidth]{obrazky-figures/tests/test45.pdf}
    %\caption{Test 45.}
    \label{test45}
\end{center}
\subsection*{Test 46}
\begin{itemize}
\item Kategorie: VÍCE PARAMETRŮ, IGNOROVÁNÍ PARAMETRŮ, NÁVRATOVÝ PARAMETR.
\item Popis: Návratový parametr je kontrolován u dvou metod targetu a parametry z obou metod jsou použity jako vstupní parametry do metody ve spoileru. Metody ve vláknu targetu $T$ jsou volány se správnou synchronizací, ale ve vláknu spoileru $S$, jsou volány bez synchronizace. 
\item Kontrakt: \texttt{X:m1() Y:m2() <- { m3(X,Y) }}

\item Očekávaný výsledek: 1 nalezených porušení.
\item Skutečný výsledek: 1 nalezených porušení.
\end{itemize}
\begin{center}
    \centering
    \includegraphics[width=\textwidth]{obrazky-figures/tests/test46.pdf}
    %\caption{Test 46.}
    \label{test46}
\end{center}
\subsection*{Test 47}
\begin{itemize}
\item Kategorie: VÍCE PARAMETRŮ, IGNOROVÁNÍ PARAMETRŮ, NÁVRATOVÝ PARAMETR.
\item Popis: Je definováno více spoilerů na jeden target s kombinací parametrů a návratového parametru. 
\item Kontrakt: \texttt{X:m1(_) m2(A, X) <- { m3(_,X) | m1(A) X:m4(A) }}
\item Očekávaný výsledek: 2 nalezených porušení.
\item Skutečný výsledek: 2 nalezených porušení.

\end{itemize}
\begin{center}
    \centering
    \includegraphics[width=\textwidth]{obrazky-figures/tests/test47.pdf}
    %\caption{Test 47.}
    \label{test47}
\end{center}
\subsection*{Test 48}
\begin{itemize}
\item Kategorie: VÍCE PARAMETRŮ, IGNOROVÁNÍ PARAMETRŮ, NÁVRATOVÝ PARAMETR.
\item Popis: Dokončená instance tageru, ve vláknu $S$, je tvořena až druhými metodami \textit{m1} a \textit{m2}, protože by jinak nesouhlasí parametry s ohledem na definici kontraktu. Totéž platí u metody \textit{m3} ve vláknu targetu $T$. 
\item Kontrakt: \texttt{X:m1(_) m2(A, X) <- { m3(_,X) | m1(A) X:m4(A) }}

\item Očekávaný výsledek: 1 nalezených porušení.
\item Skutečný výsledek: 1 nalezených porušení.
\end{itemize}
\begin{center}
    \centering
    \includegraphics[width=\textwidth]{obrazky-figures/tests/test48.pdf}
    %\caption{Test 48.}
    \label{test48}
\end{center}


\section{Testování SearchBestie}

[Poznamky z konzultace] Cim vic testu, tim lip - bud kapitola, nebo priloha. Popsat samotny test co co znamena a pak vytvorit reprezentaci jednotlivych testu a ty tam dat. Idealne diagram + kratky popis. Vstupy + vystupy a jak to dopadlo.
40 nebo i 100 ruznych testu - jak mi zbude cas
Kdyz toho bude moc, tak to bude priloha, ale musi tam byt to, jak se to cte. Vymyslet reprezentaci jak budou ty testy reprezentovany v dokumentaci. Dat tam diagram a popsat co se tam deje. (co je na vstupu)

\chapter{Závěr}

[Poznamky z konzultace] Dalsi pokracovani v projektu, sebekritiky! (co to neumi, jak by se to melo upravit aby to umelo, kde je slaba stranka, neni to optimalizovane na velke programy atd.), cim vic se priznam, ze neco neni dodelane, tim lip
Nedostatek je treba, ze se nevypousti ty metody z toho okna.

kdyz je provede target se spoilerem ve stejnem vlakne, tak to nezdetekuje
velke mnozstvi instanci muze byt problem - nezahazuji se

\chapter{SP: 1. Testování paralelních programů}

%!!!!!!!!!!!!!!!!!!!!!!!!!!!!!!!!!!!!!!!!!!!!!!!!!!!!!!!!!!!!!!!!!!!!!!!!\\
%!!!!!!!!!!!!!!!!!!!!!!!!!!!!!!!!!!!!!!!!!!!!!!!!!!!!!!!!!!!!!!!!!!!!!!!!\\
%!!!! SEMESTRALNI PROJEKT !!!!\\
%!!!!!!!!!!!!!!!!!!!!!!!!!!!!!!!!!!!!!!!!!!!!!!!!!!!!!!!!!!!!!!!!!!!!!!!!\\
%!!!!!!!!!!!!!!!!!!!!!!!!!!!!!!!!!!!!!!!!!!!!!!!!!!!!!!!!!!!!!!!!!!!!!!!!\\

%Dříve, než budou popisovány samotné dynamické analyzátory, kterými se tato práce zabývá, je nutné představit testování paralelních programů v~obecnějším pohledu. Tato kapitola se tedy zabývá obecně problematikou paralelních programů, představením základních chyb v~paralelních programech a vysvětlením některých pojmů, které budou důležité pro pochopení dalších částí této práce. Na některých místech budou uvedeny informace související s~jazykem Java, jelikož tento jazyk bude použit pro implementaci vybraných dynamických analyzátorů.

\section{Paralelní program}\label{paralel-program}
%V~dnešní době nalezneme v~běžných počítačích procesory, které mají 2 nebo více jader. S~rostoucí popularitou těchto procesorů také roste popularita paralelních programů, které tyto jádra mohou současně využívat. Využívání více jader procesoru (při běhu jednoho paralelního programu) lze dosáhnout dvěma způsoby. Prvním způsobem je vytvořit více procesů. V~tomto kontextu lze proces chápat jako sekvenčně prováděný program ve vlastním adresovém prostoru. Nevýhodou tohoto řešení je rychlost přepínání kontextu a rychlost synchronizace mezi procesy. Typickým způsobem komunikace mezi vlákny je \textit{IPC} (Inter Process Communication) \cite{cite:IPC, cite:paralel-programming}.

%Druhý způsob, je použití jednoho procesu obsahujícího několik vláken\footnote{V jazyce Java je program chápán jako proces s~vlákny, přestože lze vytvořit i programy, které budou obsahovat více procesů. Nicméně každý proces obsahuje minimálně jedno vlákno \cite{cite:java}.}. V~tomto kontextu proces chápeme jako sadu souběžně prováděných vláken v~jednom adresovém prostoru a proces je tedy obálkou těchto vláken. Toto řešení přináší několik výhod. Zejména se jedná o~rychlost přepínání kontextu a synchronizace mezi vlákny. Další výhodou je sdílení prostředků (paměť, otevřené soubory, atd.), což je umožněno díky společnému adresovému prostoru vláken stejného procesu. Přestože použití vláken přináší řadu výhod, přináší také problémy, a to právě v~podobě souběžného přístupu ke zmíněným sdíleným prostředkům. Tento přístup musí být řízen pomocí tzv. \textit{synchronizace} \cite{cite:paralel-programming}.
%Rozdíl mezi výše zmíněnými způsoby je znázorněn na obrázku \ref{}.
%...obrazek...

%Základním požadavkem v~dnešních operačních systémech\footnote{Dnešními operačními systémy jsou myšleny multiprogramové operační systémy, kde uživatel požaduje souběžný běh více programů a tedy souběžný běh více procesů a vláken.} je souběžné provádění více programů. Vzhledem k~tomu, že počet procesů a vláken, které mají běžet souběžně, je většinou vyšší než počet fyzických jader počítače, musí být výpočetní čas procesoru rozprostřen mezi všechny tyto procesy a vlákna. Rozprostření výpočetního času zajišťuje \textit{plánovač}, který plánuje přepínání kontextu procesů a vláken. Plánovač je v~běžných počítačích typicky preemptivní\footnote{Při preemptivním plánování může být procesor procesu odebrán bez jeho přičinění (ukončení procesu, zahájení čekání, atd.).} a zohledňuje několik vlastností procesů a vláken, jako je například priorita, čas strávený na procesu, doba čekání, paměťové požadavky, atd. Díky plánovači a dalším zdrojům nedeterminismu (obsluha přerušení, blokující operace, atd.) je paralelní provádění programu také nedeterministické, tj. procesy a vlákna jsou různě \textit{proloženy}. Tento nedeterminismus způsobuje, že chyby vzniklé paralelním prováděním programu se v~jednom běhu mohou vyskytnout, zatímco v~jiném ne. Z~hlediska testování paralelních programů je tedy důležité najít takové proložení, při kterém se daná chyba vyskytne \cite{cite:OS1}. 

%DAL PSAT O VLAKNECH.
%V dalším textu bude uvažováno použití jednoho procesu s více vlákny, protože právě tento způsob je nebezpečný z hlediska souběžného přístupu vláken ke sdíleným prostředkům.  



%\chapter{Návrh dynamických analyzátorů pro SearchBestie}


%\section{Analyzátor kontraktů}

%Analyzátor kontraktů, popsaný v~kapitole \ref{contract-def-ext}, bude implementovat rozšířenou definici kontraktu. Analyzátor bude implementován jako nástroj v~projektu RoadRunner, tj. hlavní třída \texttt{ContractTool} tohoto analyzátoru bude rozšiřovat třídu \texttt{Tool} nástroje \textit{RoadRunner}. Dynamický analyzátor bude redefinovat zejména následující metody:
%\begin{itemize}
%\item \texttt{create(...)}, \texttt{preJoin(...)} -- metody odchytávající události vytvoření a ukončení vlákna,
%\item \texttt{acquire(...)}, \texttt{release(...)} -- metody odchytávající události přístupu k~zámkům,
%\item \texttt{access(...)}, \texttt{volatileAccess(...)} -- metody odchytávající události přístupu k~paměti, 
%\item \texttt{enter(...)}, \texttt{exit(...)} -- metody odchytávající události vstupu a výstupu do/z metod.
%\end{itemize} 

%Dále bude tento dynamický analyzátor ukládat a využívat informace v~\textit{shadow} objektech. Do objektu \texttt{ShadowThread} bude přidáno pole obsahující sekvenci posledních vykonaných metod daným vláknem a v~objektu \texttt{ShadowVar}, uloženém v~\textit{shadow location} každého místa v~paměti, budou uloženy všechny instance targetů a spoilerů kontraktu, definovaném pro objekt uložený na tomto paměťovém místě. Instance targetů a spoilerů jednotlivých kontraktů musí podporovat metody s~parametry (a návratovou hodnotou) a proto objekty \texttt{Contract} (reprezentující zadaný kontrakt) a \texttt{Instance} (reprezentující instance targetů) budou mít níže popsanou strukturu.

%Struktura objektu \texttt{Contract} bude obsahovat následující parametry:
%udelt jako tabulku
%\begin{itemize}
%\item \texttt{objectType} -- typ objektu, nad kterým je definován kontrakt,
%\item \texttt{target} -- sekvence metod targetu,
%\item \texttt{spoilers} -- seznam sekvencí metod spoilerů.
%\end{itemize}

%Struktura objektu \texttt{Instance} bude obsahovat následující parametry:
%\begin{itemize}
%\item \texttt{pattern} -- odkaz na target/spoiler, jehož je tento objekt instancí,
%\item \texttt{methodSequence} -- sekvence metod targetu/spoileru, které již byly zpozorovány,
%\item \texttt{position} -- pozice v~targetu/spoileru,
%\item \texttt{parameters} -- mapa parametrů v~metodách. Klíčem je definovaný název v~kontraktu a hodnotou dosazená hodnota při výskytu události.
%\end{itemize}

%Ve výše zmíněných objektech se vyskytují seznamy metod. Každá tato metoda bude reprezentována třídou \texttt{method}, která bude obsahovat mimo jiné názvy parametrů, které jsou uvedeny při definici kontraktu a které poté budou vytvářet klíče v~mapě \texttt{parameters} v~instancích targetů a~spoilerů.

%V~dynamickém analyzátoru budou využívány hlavně metody \texttt{enter(...)} a \texttt{exit(...)}. V~první jmenované metodě bude zjištěno, na jakém objektu byla metoda vykonána a bude provedena operace \textit{advance} na všech instancích targetu a spoilerů uložených v~\textit{shadow location} daného objektu. V~metodě \texttt{exit(...)} bude testováno, zda mezi instancemi targetů a spoilerů nedošlo, nebo nemohlo dojít, k~porušení kontraktu podle pravidel definovaných v~kapitole \ref{contract-hb}.

%%%%%%%%%%%%%%%%%%%%%%%%%%%%%%%%%%%%%%%%%%%%%%%%%%%%%%%%%%%%%%%%%%%%%%%%%%%%%%%%%%%%%%%%%%%%%%%%%%%%%%%%%%%%%%5
%%%%%%%%%%%%%%%%%%%%%%%%%%%%%%%%%%%%%%%%%%%%%%%%%%%%%%%%%%%%%%%%%%%%%%%%%%%%%%%%%%%%%%%%%%%%%%%%%%%%%%%%%%%%%555
%%%%%%%%%%%%%%%%%%%%%%%%%%%%%%%%%%%%%%%%%%%%%%%%%%%%%%%%%%%%%%%%%%%%%%%%%%%%%%%%%%%%%%%%%%%%%%%%%%%%%%%%%%%%%%%%
%%%%%%%%%%%%%%%%%%%%%%%%%%%%%%%%%%%%%%%%%%%%%%%%%%%%%%%%%%%%%%%%%%%%%%%%%%%%%%%%%%%%%%%%%%%%%%%%%%%%%%%%%%%%%%%%
%%%%%%%%%%%%%%%%%%%%%%%%%%%%%%%%%%%%%%%%%%%%%%%%%%%%%%%%%%%%%%%%%%%%%%%%%%%%%%%%%%%%%%%%%%%%%%%%%%%%%%%%%%%%%%%%
%%%%%%%%%%%%%%%%%%%%%%%%%%%%%%%%%%%%%%%%%%%%%%%%%%%%%%%%%%%%%%%%%%%%%%%%%%%%%%%%%%%%%%%%%%%%%%%%%%%%%%%%%%%%%%%%

%\chapter{Úvod -- VZOR}
% Abychom mohli napsat odborný text jasně a~srozumitelně, musíme splnit několik základních předpokladů:
% \begin{itemize}
% \item Musíme mít co říci,
% \item musíme vědět, komu to chceme říci,
% \item musíme si dokonale promyslet obsah,
% \item musíme psát strukturovaně. 
% \end{itemize}

% Tyto a další pokyny jsou dostupné též na školních internetových stránkách \cite{fitWeb}.

% Přehled základů typografie a tvorby dokumentů s využitím systému \LaTeX je 
% uveden v~\cite{Rybicka}.

% \section{Musíme mít co říci}
% Dalším důležitým předpokladem dobrého psaní je {\it psát pro někoho}. Píšeme-li si poznámky sami pro sebe, píšeme je jinak než výzkumnou zprávu, článek, diplomovou práci, knihu nebo dopis. Podle předpokládaného čtenáře se rozhodneme pro způsob psaní, rozsah informace a~míru detailů.

% \section{Musíme vědět, komu to chceme říci}
% Dalším důležitým předpokladem dobrého psaní je psát pro někoho. Píšeme-li si poznámky sami pro sebe, píšeme je jinak než výzkumnou zprávu, článek, diplomovou práci, knihu nebo dopis. Podle předpokládaného čtenáře se rozhodneme pro způsob psaní, rozsah informace a~míru detailů.

% \section{Musíme si dokonale promyslet obsah}
% Musíme si dokonale promyslet a~sestavit obsah sdělení a~vytvořit pořadí, v~jakém chceme čtenáři své myšlenky prezentovat. 
% Jakmile víme, co chceme říci a~komu, musíme si rozvrhnout látku. Ideální je takové rozvržení, které tvoří logicky přesný a~psychologicky stravitelný celek, ve kterém je pro všechno místo a~jehož jednotlivé části do sebe přesně zapadají. Jsou jasné všechny souvislosti a~je zřejmé, co kam patří.

% Abychom tohoto cíle dosáhli, musíme pečlivě organizovat látku. Rozhodneme, co budou hlavní kapitoly, co podkapitoly a~jaké jsou mezi nimi vztahy. Diagramem takové organizace je graf, který je velmi podobný stromu, ale ne řetězci. Při organizaci látky je stejně důležitá otázka, co do osnovy zahrnout, jako otázka, co z~ní vypustit. Příliš mnoho podrobností může čtenáře právě tak odradit jako žádné detaily.

% Výsledkem této etapy je osnova textu, kterou tvoří sled hlavních myšlenek a~mezi ně zařazené detaily.

% \section{Musíme psát strukturovaně} 
% Musíme začít psát strukturovaně a~současně pracujeme na co nejsrozumitelnější formě, včetně dobrého slohu a~dokonalého značení. 
% Máme-li tedy myšlenku, představu o~budoucím čtenáři, cíl a~osnovu textu, můžeme začít psát. Při psaní prvního konceptu se snažíme zaznamenat všechny své myšlenky a~názory vztahující se k~jednotlivým kapitolám a~podkapitolám. Každou myšlenku musíme vysvětlit, popsat a~prokázat. Hlavní myšlenku má vždy vyjadřovat hlavní věta a~nikoliv věta vedlejší.

% I k~procesu psaní textu přistupujeme strukturovaně. Současně s~tím, jak si ujasňujeme strukturu písemné práce, vytváříme kostru textu, kterou postupně doplňujeme. Využíváme ty prostředky DTP programu, které podporují strukturovanou stavbu textu (předdefinované typy pro nadpisy a~bloky textu). 


% \chapter{Několik formálních pravidel}
% Naším cílem je vytvořit jasný a~srozumitelný text. Vyjadřujeme se proto přesně, píšeme dobrou češtinou (nebo zpravidla angličtinou) a~dobrým slohem podle obecně přijatých zvyklostí. Text má upravit čtenáři cestu k~rychlému pochopení problému, předvídat jeho obtíže a~předcházet jim. Dobrý sloh předpokládá bezvadnou gramatiku, správnou interpunkci a~vhodnou volbu slov. Snažíme se, aby náš text nepůsobil příliš jednotvárně používáním malého výběru slov a~tím, že některá zvlášť oblíbená slova používáme příliš často. Pokud používáme cizích slov, je samozřejmým předpokladem, že známe jejich přesný význam. Ale i~českých slov musíme používat ve správném smyslu. Např. platí jistá pravidla při používání slova {\it zřejmě}. Je {\it zřejmé} opravdu zřejmé? A~přesvědčili jsme se, zda to, co je {\it zřejmé} opravdu platí? Pozor bychom si měli dát i~na příliš časté používání zvratného se. Například obratu {\it dokázalo se}, že... zásadně nepoužíváme. Není špatné používat autorského {\it my}, tím předpokládáme, že něco řešíme, nebo například zobecňujeme spolu se čtenářem. V~kvalifikačních pracích použijeme autorského {\it já} (například když vymezujeme podíl vlastní práce vůči převzatému textu), ale v~běžném textu se nadměrné používání první osoby jednotného čísla nedoporučuje.

% Za pečlivý výběr stojí i~symbolika, kterou používáme ke {\it značení}. Máme tím na mysli volbu zkratek a~symbolů používaných například pro vyjádření typů součástek, pro označení hlavních činností programu, pro pojmenování ovládacích kláves na klávesnici, pro pojmenování proměnných v~matematických formulích a~podobně. Výstižné a~důsledné značení může čtenáři při četbě textu velmi pomoci. Je vhodné uvést seznam značení na začátku textu. Nejen ve značení, ale i~v~odkazech a~v~celkové tiskové úpravě je důležitá důslednost.

% S tím souvisí i~pojem z~typografie nazývaný {\it vyznačování}. Zde máme na mysli způsob sazby textu pro jeho zvýraznění. Pro zvolené značení by měl být zvolen i~způsob vyznačování v~textu. Tak například klávesy mohou být umístěny do obdélníčku, identifikátory ze zdrojového textu mohou být vypisovány {\tt písmem typu psací stroj} a~podobně.

% Uvádíme-li některá fakta, neskrýváme jejich původ a~náš vztah k~nim. Když něco tvrdíme, vždycky výslovně uvedeme, co z~toho bylo dokázáno, co teprve bude dokázáno v~našem textu a~co přebíráme z~literatury s~uvedením odkazu na příslušný zdroj. V~tomto směru nenecháváme čtenáře nikdy na pochybách, zda jde o~myšlenku naši nebo převzatou z~literatury.

% Nikdy neplýtváme čtenářovým časem výkladem triviálních a~nepodstatných informací. Neuvádíme rovněž několikrát totéž jen jinými slovy. Při pozdějších úpravách textu se nám může některá dříve napsaná pasáž jevit jako zbytečně podrobná nebo dokonce zcela zbytečná. Vypuštění takové pasáže nebo alespoň její zestručnění přispěje k~lepší čitelnosti práce! Tento krok ale vyžaduje odvahu zahodit čas, který jsme jejímu vytvoření věnovali. 


% \chapter{Nikdy to nebude naprosto dokonalé}
% Když jsme už napsali vše, o~čem jsme přemýšleli, uděláme si den nebo dva dny volna a~pak si přečteme sami rukopis znovu. Uděláme ještě poslední úpravy a~skončíme. Jsme si vědomi toho, že vždy zůstane něco nedokončeno, vždy existuje lepší způsob, jak něco vysvětlit, ale každá etapa úprav musí být konečná.


% \chapter{Typografické a~jazykové zásady}
% Při tisku odborného textu typu {\it technická zpráva} (anglicky {\it technical report}), ke kterému patří například i~text kvalifikačních prací, se často volí formát A4 a~často se tiskne pouze po jedné straně papíru. V~takovém případě volte levý okraj všech stránek o~něco větší než pravý -- v~tomto místě budou papíry svázány a~technologie vazby si tento požadavek vynucuje. Při vazbě s~pevným hřbetem by se levý okraj měl dělat o~něco širší pro tlusté svazky, protože se stránky budou hůře rozevírat a~levý okraj se tak bude oku méně odhalovat.

% Horní a~spodní okraj volte stejně veliký, případně potištěnou část posuňte mírně nahoru (horní okraj menší než dolní). Počítejte s~tím, že při vazbě budou okraje mírně oříznuty.

% Pro sazbu na stránku formátu A4 je vhodné používat pro základní text písmo stupně (velikosti) 11 bodů. Volte šířku sazby 15 až 16 centimetrů a~výšku 22 až 23 centimetrů (včetně případných hlaviček a~patiček). Proklad mezi řádky se volí 120 procent stupně použitého základního písma, což je optimální hodnota pro rychlost čtení souvislého textu. V~případě použití systému LaTeX ponecháme implicitní nastavení. Při psaní kvalifikační práce se řiďte příslušnými závaznými požadavky.

% Stupeň písma u~nadpisů různé úrovně volíme podle standardních typografických pravidel. 
% Pro všechny uvedené druhy nadpisů se obvykle používá polotučné nebo tučné písmo (jednotně buď všude polotučné nebo všude tučné). Proklad se volí tak, aby se následující text běžných odstavců sázel pokud možno na {\it pevný rejstřík}, to znamená jakoby na linky s~předem definovanou a~pevnou roztečí.

% Uspořádání jednotlivých částí textu musí být přehledné a~logické. Je třeba odlišit názvy kapitol a~podkapitol -- píšeme je malými písmeny kromě velkých začátečních písmen. U~jednotlivých odstavců textu odsazujeme první řádek odstavce asi o~jeden až dva čtverčíky (vždy o~stejnou, předem zvolenou hodnotu), tedy přibližně o~dvě šířky velkého písmene M základního textu. Poslední řádek předchozího odstavce a~první řádek následujícího odstavce se v~takovém případě neoddělují svislou mezerou. Proklad mezi těmito řádky je stejný jako proklad mezi řádky uvnitř odstavce.

% Při vkládání obrázků volte jejich rozměry tak, aby nepřesáhly oblast, do které se tiskne text (tj. okraje textu ze všech stran). Pro velké obrázky vyčleňte samostatnou stránku. Obrázky nebo tabulky o~rozměrech větších než A4 umístěte do písemné zprávy formou skládanky všité do přílohy nebo vložené do záložek na zadní desce.

% Obrázky i~tabulky musí být pořadově očíslovány. Číslování se volí buď průběžné v~rámci celého textu, nebo -- což bývá praktičtější -- průběžné v~rámci kapitoly. V~druhém případě se číslo tabulky nebo obrázku skládá z~čísla kapitoly a~čísla obrázku/tabulky v~rámci kapitoly -- čísla jsou oddělena tečkou. Čísla podkapitol nemají na číslování obrázků a~tabulek žádný vliv.

% Tabulky a~obrázky používají své vlastní, nezávislé číselné řady. Z toho vyplývá, že v~odkazech uvnitř textu musíme kromě čísla udat i~informaci o~tom, zda se jedná o~obrázek či tabulku (například ``... {\it viz tabulka 2.7} ...''). Dodržování této zásady je ostatně velmi přirozené.

% Pro odkazy na stránky, na čísla kapitol a~podkapitol, na čísla obrázků a~tabulek a~v~dalších podobných příkladech využíváme speciálních prostředků DTP programu, které zajistí vygenerování správného čísla i~v~případě, že se text posune díky změnám samotného textu nebo díky úpravě parametrů sazby. Příkladem takového prostředku v~systému LaTeX je odkaz na číslo odpovídající umístění značky v~textu, například návěští ($\backslash${\tt ref\{navesti\}} -- podle umístění návěští se bude jednat o~číslo kapitoly, podkapitoly, obrázku, tabulky nebo podobného číslovaného prvku), na stránku, která obsahuje danou značku ($\backslash${\tt pageref\{navesti\}}), nebo na literární odkaz ($\backslash${\tt cite\{identifikator\}}).

% Rovnice, na které se budeme v~textu odvolávat, opatříme pořadovými čísly při pravém okraji příslušného řádku. Tato pořadová čísla se píší v~kulatých závorkách. Číslování rovnic může být průběžné v~textu nebo v~jednotlivých kapitolách.

% Jste-li na pochybách při sazbě matematického textu, snažte se dodržet způsob sazby definovaný systémem LaTeX. Obsahuje-li vaše práce velké množství matematických formulí, doporučujeme dát přednost použití systému LaTeX.

% Mezeru neděláme tam, kde se spojují číslice s~písmeny v~jedno slovo nebo v~jeden znak -- například {\it 25krát}.

% Členicí (interpunkční) znaménka tečka, čárka, středník, dvojtečka, otazník a~vykřičník, jakož i~uzavírací závorky a~uvozovky se přimykají k~předcházejícímu slovu bez mezery. Mezera se dělá až za nimi. To se ovšem netýká desetinné čárky (nebo desetinné tečky). Otevírací závorka a~přední uvozovky se přimykají k~následujícímu slovu a~mezera se vynechává před nimi -- (takto) a~``takto''.

% Pro spojovací a~rozdělovací čárku a~pomlčku nepoužíváme stejný znak. Pro pomlčku je vyhrazen jiný znak (delší). V~systému TeX (LaTeX) se spojovací čárka zapisuje jako jeden znak ``pomlčka'' (například ``Brno-město''), pro sázení textu ve smyslu intervalu nebo dvojic, soupeřů a~podobně se ve zdrojovém textu používá dvojice znaků ``pomlčka'' (například ``zápas Sparta -- Slavie''; ``cena 23--25 korun''), pro výrazné oddělení části věty, pro výrazné oddělení vložené věty, pro vyjádření nevyslovené myšlenky a~v~dalších situacích (viz Pravidla českého pravopisu) se používá nejdelší typ pomlčky, která se ve zdrojovém textu zapisuje jako trojice znaků ``pomlčka'' (například ``Další pojem --- jakkoliv se může zdát nevýznamný --- bude neformálně definován v~následujícím odstavci.''). Při sazbě matematického mínus se při sazbě používá rovněž odlišný znak. V~systému TeX je ve zdrojovém textu zapsán jako normální mínus (tj. znak ``pomlčka''). Sazba v~matematickém prostředí, kdy se vzoreček uzavírá mezi dolary, zajistí vygenerování správného výstupu.

% Lomítko se píše bez mezer. Například školní rok 2008/2009.

% Pravidla pro psaní zkratek jsou uvedena v~Pravidlech českého pravopisu \cite{Pravidla}. I~z~jiných důvodů je vhodné, abyste tuto knihu měli po ruce. 


% \section{Co to je normovaná stránka?}
% Pojem {\it normovaná stránka} se vztahuje k~posuzování objemu práce, nikoliv k~počtu vytištěných listů. Z historického hlediska jde o~počet stránek rukopisu, který se psal psacím strojem na speciální předtištěné formuláře při dodržení průměrné délky řádku 60 znaků a~při 30 řádcích na stránku rukopisu. Vzhledem k~zápisu korekturních značek se používalo řádkování 2 (ob jeden řádek). Tyto údaje (počet znaků na řádek, počet řádků a~proklad mezi nimi) se nijak nevztahují ke konečnému vytištěnému výsledku. Používají se pouze pro posouzení rozsahu. Jednou normovanou stránkou se tedy rozumí 60*30 = 1800 znaků. Obrázky zařazené do textu se započítávají do rozsahu písemné práce odhadem jako množství textu, které by ve výsledném dokumentu potisklo stejně velkou plochu.

% Orientační rozsah práce v~normostranách lze v~programu Microsoft Word zjistit pomocí funkce {\it Počet slov} v~menu {\it Nástroje}, když hodnotu {\it Znaky (včetně mezer)} vydělíte konstantou 1800. Do rozsahu práce se započítává pouze text uvedený v~jádru práce. Části jako abstrakt, klíčová slova, prohlášení, obsah, literatura nebo přílohy se do rozsahu práce nepočítají. Je proto nutné nejdříve označit jádro práce a~teprve pak si nechat spočítat počet znaků. Přibližný rozsah obrázků odhadnete ručně. Podobně lze postupovat i~při použití OpenOffice. Při použití systému LaTeX pro sazbu je situace trochu složitější. Pro hrubý odhad počtu normostran lze využít součet velikostí zdrojových souborů práce podělený konstantou cca 2000 (normálně bychom dělili konstantou 1800, jenže ve zdrojových souborech jsou i~vyznačovací příkazy, které se do rozsahu nepočítají). Pro přesnější odhad lze pak vyextrahovat holý text z~PDF (např. metodou cut-and-paste nebo {\it Save as Text...}) a~jeho velikost podělit konstantou 1800. 


% \chapter{Závěr}
% Závěrečná kapitola obsahuje zhodnocení dosažených výsledků se zvlášť vyznačeným vlastním přínosem studenta. Povinně se zde objeví i zhodnocení z pohledu dalšího vývoje projektu, student uvede náměty vycházející ze zkušeností s řešeným projektem a uvede rovněž návaznosti na právě dokončené projekty.

%=========================================================================
%@INPROCEEDINGS{cite:contract2,
%   author = {Carla Ferreira and Jan Fiedor and Joao Lourenco and Aleš Smrčka and Diogo G. Sousa and
%	Tomáš Vojnar},
%   title = {\it {Verifying Concurrent Programs Using Contracts}},
%   booktitle = {Proceedings of ICST Conference},
%   year = {2017},
%   location = {Brno University of Technology, CZ},
%   language = {english}
%} % viz. obsah.tex / see obsah.tex

  % Pouzita literatura / Bibliography
  % ----------------------------------------------
\ifslovak
  \makeatletter
  \def\@openbib@code{\addcontentsline{toc}{chapter}{Literatúra}}
  \makeatother
  \bibliographystyle{bib-styles/czechiso}
\else
  \ifczech
    \makeatletter
    \def\@openbib@code{\addcontentsline{toc}{chapter}{Literatura}}
    \makeatother
    \bibliographystyle{bib-styles/czechiso}
  \else 
    \makeatletter
    \def\@openbib@code{\addcontentsline{toc}{chapter}{Bibliography}}
    \makeatother
    \bibliographystyle{bib-styles/englishiso}
  %  \bibliographystyle{alpha}
  \fi
\fi
  \begin{flushleft}
  \bibliography{xjanou14-dynamicke-analyzatory-20-literatura-bibliography}
  \end{flushleft}

  % vynechani stranky v oboustrannem rezimu
  % Skip the page in the two-sided mode
  \iftwoside
    \cleardoublepage
  \fi

  % Prilohy / Appendices
  % ---------------------------------------------
  \appendix
\ifczech
  \renewcommand{\appendixpagename}{Přílohy}
  \renewcommand{\appendixtocname}{Přílohy}
  \renewcommand{\appendixname}{Příloha}
\fi
\ifslovak
  \renewcommand{\appendixpagename}{Prílohy}
  \renewcommand{\appendixtocname}{Prílohy}
  \renewcommand{\appendixname}{Príloha}
\fi
  %\appendixpage

% vynechani stranky v oboustrannem rezimu
% Skip the page in the two-sided mode
\iftwoside
  \cleardoublepage
\fi
  
\ifslovak
%  \section*{Zoznam príloh}
%  \addcontentsline{toc}{section}{Zoznam príloh}
\else
  \ifczech
%    \section*{Seznam příloh}
%    \addcontentsline{toc}{section}{Seznam příloh}
  \else
%    \section*{List of Appendices}
%    \addcontentsline{toc}{section}{List of Appendices}
  \fi
\fi
  \startcontents[chapters]
  % seznam příloh / list of appendices
  % \printcontents[chapters]{l}{0}{\setcounter{tocdepth}{2}}
  
  % vynechani stranky v oboustrannem rezimu
  \iftwoside
    \cleardoublepage
  \fi
  % Prilohy
  %\input{xjanou14-dynamicke-analyzatory-30-prilohy-appendices} % viz. prilohy.tex / see prilohy.tex
\end{document}
